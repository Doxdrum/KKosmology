\usepackage{mathtools,amsmath,amssymb,amsfonts,dsfont,mathrsfs,amsthm,bm,latexsym}
\usepackage{graphicx}
\usepackage{centernot}
\usepackage{xcolor}
\usepackage{comment}
%% \usepackage{hyperref}
%% \hypersetup{linktocpage,colorlinks=true,urlcolor=blue!80!red,linkcolor=blue,citecolor=red}
\usepackage{feynmf}
\usepackage{siunitx}
\usepackage{array}
\newcolumntype{L}[1]{>{\raggedright\let\newline\\\arraybackslash\hspace{0pt}}m{#1}}
\newcolumntype{C}[1]{>{\centering\let\newline\\\arraybackslash\hspace{0pt}}m{#1}}
\newcolumntype{R}[1]{>{\raggedleft\let\newline\\\arraybackslash\hspace{0pt}}m{#1}}
%% \usepackage{ulem}
\usepackage{tikz}
\usepackage{braket}
\usepackage{xparse}
\usetikzlibrary{shapes,
  snakes,
  decorations.pathmorphing,
  decorations.markings,
  calc,
  shadows.blur,
  shadings}
\usepackage[framemethod=tikz]{mdframed}

\makeatletter
% gluon decoration (based on the original coil decoration)
\pgfdeclaredecoration{gluon}{coil}
{
  \state{coil}[switch if less than=%
    0.5\pgfdecorationsegmentlength+%>
    \pgfdecorationsegmentaspect\pgfdecorationsegmentamplitude+%
    \pgfdecorationsegmentaspect\pgfdecorationsegmentamplitude to last,
               width=+\pgfdecorationsegmentlength]
  {
    \pgfpathcurveto
    {\pgfpoint@oncoil{0    }{ 0.555}{1}}
    {\pgfpoint@oncoil{0.445}{ 1    }{2}}
    {\pgfpoint@oncoil{1    }{ 1    }{3}}
    \pgfpathcurveto
    {\pgfpoint@oncoil{1.555}{ 1    }{4}}
    {\pgfpoint@oncoil{2    }{ 0.555}{5}}
    {\pgfpoint@oncoil{2    }{ 0    }{6}}
    \pgfpathcurveto
    {\pgfpoint@oncoil{2    }{-0.555}{7}}
    {\pgfpoint@oncoil{1.555}{-1    }{8}}
    {\pgfpoint@oncoil{1    }{-1    }{9}}
    \pgfpathcurveto
    {\pgfpoint@oncoil{0.445}{-1    }{10}}
    {\pgfpoint@oncoil{0    }{-0.555}{11}}
    {\pgfpoint@oncoil{0    }{ 0    }{12}}
  }
  \state{last}[next state=final]
  {
    \pgfpathcurveto
    {\pgfpoint@oncoil{0    }{ 0.555}{1}}
    {\pgfpoint@oncoil{0.445}{ 1    }{2}}
    {\pgfpoint@oncoil{1    }{ 1    }{3}}
    \pgfpathcurveto
    {\pgfpoint@oncoil{1.555}{ 1    }{4}}
    {\pgfpoint@oncoil{2    }{ 0.555}{5}}
    {\pgfpoint@oncoil{2    }{ 0    }{6}}
  }
  \state{final}{}
}

\def\pgfpoint@oncoil#1#2#3{%
  \pgf@x=#1\pgfdecorationsegmentamplitude%
  \pgf@x=\pgfdecorationsegmentaspect\pgf@x%
  \pgf@y=#2\pgfdecorationsegmentamplitude%
  \pgf@xa=0.083333333333\pgfdecorationsegmentlength%
  \advance\pgf@x by#3\pgf@xa%
}
\makeatother

\tikzset{
  boson/.style={decorate,decoration={gluon,segment length=9pt,aspect=0}},
  % style to apply some styles to each segment of a path
  on each segment/.style={
    decorate,
    decoration={
      show path construction,
      moveto code={},
      lineto code={
        \path [#1]
        (\tikzinputsegmentfirst) -- (\tikzinputsegmentlast);
      },
      curveto code={
        \path [#1] (\tikzinputsegmentfirst)
        .. controls
        (\tikzinputsegmentsupporta) and (\tikzinputsegmentsupportb)
        ..
        (\tikzinputsegmentlast);
      },
      closepath code={
        \path [#1]
        (\tikzinputsegmentfirst) -- (\tikzinputsegmentlast);
      },
    },
  },
  % style to add an arrow in the middle of a path
  mid arrow/.style={postaction={decorate,decoration={
        markings,
        mark=at position .5 with {\arrow[#1]{stealth}}
      }}},
}
\tikzstyle fermion=[draw,postaction={on each segment={mid arrow}}]


%-------------------------------Theorems
\newtheorem{Def}{Definition}[section]
\newtheorem{Thm}[Def]{Theorem}
\newtheorem{Lem}[Def]{Lemma}
\newtheorem{Pos}[Def]{Postulate}
\newtheorem{Exa}[Def]{Example}
\newtheorem{Cor}[Def]{Corrolary}
\newtheorem{Pro}[Def]{Proposition}

%---------------------------------New commands
\newcommand{\A}{\mathcal{A}} %% This is equivalent to \Ag (no args)
\newcommand{\abs}[1]{\left|{#1}\right|}
\DeclareMathOperator{\ad}{ad}
\DeclareMathOperator{\Ad}{Ad}
\DeclareDocumentCommand{\Ag}{ s o }{ \IfBooleanTF{#1}
    { \IfValueTF{#2}{ \bm{\mathcal{A}}_{(#2)} }{ \bm{\mathcal{A}} } }
    { \IfValueTF{#2}{    {\mathcal{A}}_{(#2)} }{    {\mathcal{A}} } } }
%% \newcommand{\Ag}{\mathcal{A}_{(1)}}
%% \newcommand{\Agf}{\boldsymbol{\mathcal{A}}}
\DeclareDocumentCommand{\Af}{ s o }{ \IfBooleanTF{#1}
    { \IfValueTF{#2}{ \boldsymbol{A}_{(#2)} }{ \boldsymbol{A} } }
    { \IfValueTF{#2}{            {A}_{(#2)} }{            {A} } } }
%% \newcommand{\Af}{ {\mathbf{A}} }
%% \newcommand{\AF}[1]{ {\mathbf{A}}_{({#1})} }
\newcommand{\hAf}{ \hat{\mathbf{A}}_{(1)} }
\newcommand{\hAF}[1]{ \hat{\mathbf{A}}_{(#1)} }
\newcommand{\bboxed}[1]{{\color{red}{\boxed{\boxed{\textcolor{black}{#1}}}}}}
\newcommand{\C}{\mathbb{C}}
\newcommand{\Cl}{\mathcal{C}\!\ell}
\newcommand{\cdf}[1][]{{\boldsymbol{\mathcal{D}}}{#1}}
\newcommand{\covd}{\mathcal{D}}
\newcommand{\D}{\mathscr{D}}
\newcommand{\df}[1][]{\mathbf{d}{#1}}
\newcommand{\dfd}{{\mathbf{d}}^\dag\!}
\newcommand\ded{\mathrm{d}^\dag}
\DeclareMathOperator{\End}{End}
\newcommand{\ele}[2][]{\frac{d}{dt}\left(\frac{\partial\mathcal{L}}{\partial \dot{#2}^{#1}}\right) - \frac{\partial\mathcal{L}}{\partial {#2}^{#1}}}
\newcommand{\fele}[2][]{\partial_\mu \left(\frac{\delta\mathcal{L}}{\delta\left(\partial_\mu{#2}^{#1}\right)}\right) - \frac{\delta\mathcal{L}}{\delta {#2}^{#1}}}
\newcommand{\vb}[1]{\vec{e}_{#1}}
%% \newcommand{\fb}[1]{\widetilde{e}{}^{ #1}}
\newcommand{\fb}[1]{\widetilde{e}{}^{ #1}}
\newcommand\fder[3][]{\frac{\delta^{#1}{#2}}{\delta {#3}^{#1}}}
\newcommand\fdern[4][]{\frac{\delta^{#1}{#2}}{\delta {#3} \cdots \delta {#4}}}
%% \newcommand{\Fc}{\mathcal{F}}
%% \newcommand{\F}{\boldsymbol{\mathcal{F}}}
%% \newcommand{\Fg}{\boldsymbol{\mathcal{F}}_{(2)}}
\DeclareDocumentCommand{\Fg}{ s o }{ \IfBooleanTF{#1}
    { \IfValueTF{#2}{ \bm{\mathcal{F}}_{(#2)} }{ \bm{\mathcal{F}} } }
    { \IfValueTF{#2}{    {\mathcal{F}}_{(#2)} }{    {\mathcal{F}} } } }
%% \newcommand{\Ff}{{\mathbf{F}}}
\DeclareDocumentCommand{\Ff}{ s o }{ \IfBooleanTF{#1}
    { \IfValueTF{#2}{ \boldsymbol{F}_{(#2)} }{ \boldsymbol{F} } }
    { \IfValueTF{#2}{            {F}_{(#2)} }{            {F} } } }
%% \DeclareDocumentCommand{\Ff}{ o }{ \IfValueTF{#1}{ F_{(#1)} }{ F } }
\newcommand{\FF}[1]{{\mathbf{F}}_{(#1)}}
\newcommand{\hFF}[1]{{\mathbf{\hat{F}}}_{(#1)}}
\newcommand{\fy}{\centernot}
\newcommand{\G}{\mathscr{G}}
\newcommand{\ga}{\gamma}
\newcommand{\gf}{\boldsymbol{\gamma}}
\newcommand{\Ga}{\Gamma}
\newcommand{\Ha}{\mathscr{H}}
\newcommand{\He}{\mathbb{H}}
\newcommand{\Hi}{\mathcal{H}}
\newcommand{\Hint}{\underline{\sc Hint:} }
\DeclareMathOperator{\hs}{\star\!\!}
\DeclareMathOperator{\st}{\star}
\newcommand{\J}{\mathscr{J}}
\newcommand{\K}{\mathbb{K}}
\newcommand{\KK}{Kaluza--Klein\ }
\newcommand{\Lag}{\mathscr{L}}
\newcommand{\Li}{\mathcal{L}}
\newcommand{\La}[1][]{\triangle_{#1}}
\newcommand{\Lap}{\nabla^2}
\newcommand{\lr}[1]{\stackrel{\leftrightarrow}{#1}}
\newcommand{\M}{\ensuremath{\mathscr{M}}}
\DeclareMathOperator{\Mat}{Mat}
\newcommand{\Mi}{\mathcal{M}}
\newcommand{\MN}{Maldacena-N\'u\~nez\ }
\newcommand{\N}{\ensuremath{\mathscr{N}}}
\newcommand{\Na}{\mathbb{N}}
\newcommand{\No}{\mathcal{N}}
\newcommand{\norm}[1]{\left\|#1\right\|}
\newcommand{\Op}{\mathcal{O}}
\newcommand{\Or}{\mathscr{O}}
\DeclareDocumentCommand{\PB}{ O{m} O{q} O{p} m m }{ \frac{ \partial #4 }{\partial {#2}^{#1} } \frac{ \partial #5 }{\partial {#3}_{#1} } - \frac{ \partial #4 }{\partial {#3}_{#1} } \frac{ \partial #5 }{\partial {#2}^{#1} } }
\newcommand\pder[3][]{\frac{\partial^{#1}{#2}}{\partial {#3}^{#1}}}
\newcommand\pdern[4][]{\frac{\partial^{#1}#2}{\partial #3\cdots\partial #4}}
\DeclareMathOperator{\Pfaff}{Pfaff}
\newcommand{\Qh}[1][]{\ensuremath{\hat{Q}_{#1}}}
\newcommand{\R}{\mathbb{R}}
\newcommand{\Ri}{\mathcal{R}}
%% \newcommand{\S}{\mathscr{S}}
\newcommand{\SM}{Standard Model {}}%\mathscr{S}}
\DeclareDocumentCommand\Te{o o m }{\mathcal{T}{}^{#1}_{#2}(#3)}
\newcommand{\T}{\mathscr{T}}
\newcommand\vdj[1]{\left< \Delta J \right>_{#1}}
\newcommand{\w}{{\scriptstyle\wedge}\!}
\newcommand{\we}{{\scriptstyle\wedge}}
\newcommand{\Z}{\mathbb{Z}}

\newcommand{\dbar}[1]{\ensuremath{\mathchar'26\mkern-12mu \mathrm{d}^{#1}}\!}


\DeclareDocumentCommand{\BM}{ s }{ \IfBooleanTF{#1} {\hat{\bm{M}}}{\bm{M}} }
\DeclareDocumentCommand{\BN}{ s }{ \IfBooleanTF{#1} {\hat{\bm{N}}}{\bm{N}} }
\DeclareDocumentCommand{\BP}{ s }{ \IfBooleanTF{#1} {\hat{\bm{P}}}{\bm{P}} }
\DeclareDocumentCommand{\BQ}{ s }{ \IfBooleanTF{#1} {\hat{\bm{Q}}}{\bm{Q}} }
\DeclareDocumentCommand{\BR}{ s }{ \IfBooleanTF{#1} {\hat{\bm{R}}}{\bm{R}} }
\DeclareDocumentCommand{\BS}{ s }{ \IfBooleanTF{#1} {\hat{\bm{S}}}{\bm{S}} }
\DeclareDocumentCommand{\BU}{ s }{ \IfBooleanTF{#1} {\hat{\bm{U}}}{\bm{U}} }
\DeclareDocumentCommand{\BV}{ s }{ \IfBooleanTF{#1} {\hat{\bm{V}}}{\bm{V}} }

%--------------------------- New Greek
\newcommand{\tht}{\ensuremath{\theta}}
\newcommand{\bet}{\ensuremath{\bar{\eta}}}
\newcommand{\bps}{\ensuremath{\bar{\psi}}}
\newcommand{\bc}{\ensuremath{\bar{\chi}}}
\newcommand{\Bps}{\ensuremath{\bar{\Psi}}}
\newcommand{\Bx}{\ensuremath{\bar{\Xi}}}
\newcommand{\bph}{\ensuremath{\bar{\phi}}}
\newcommand{\vph}{\ensuremath{\varphi}}
\newcommand{\bvph}{\ensuremath{\bar{\varphi}}}
\newcommand{\bth}{\ensuremath{\bar{\theta}}}
\newcommand{\hph}{\ensuremath{\hat{\phi}}}
\newcommand\bml{\bm{\lambda}}
\newcommand\bmk{\bm{\kappa}}

\newcommand{\bs}[1]{\boldsymbol{#1}}


\renewcommand{\div}{{\mathbf{div}}}
\newcommand{\grad}{{\mathbf{grad}}}
\newcommand{\curl}{{\mathbf{curl}}}

%%---------------------------------------------
%%--------- Vielbein definitions
%%---------------------------------------------
\NewDocumentCommand\MyAc{ m }{#1}
%%------------------------------ Vielbein form
\DeclareDocumentCommand{\vif}{ t. t, t- s s m }{
  \RenewDocumentCommand\MyAc{ m }{##1}
  \IfBooleanT{#1}{\RenewDocumentCommand\MyAc{ m }{ \mathring{##1} } }
  \IfBooleanT{#2}{\RenewDocumentCommand\MyAc{ m }{ \tilde{##1} } }
  \IfBooleanT{#3}{\RenewDocumentCommand\MyAc{ m }{ \bar{##1} } }
  \IfBooleanTF{#4}
  { \IfBooleanTF{#5} { \hat{\MyAc{\boldsymbol{e}}}^{\hat{#6}} }{ \hat{\MyAc{\boldsymbol{e}}}^{{#6}} } }
  { \MyAc{\boldsymbol{e}}^{{#6}} } }
%%------------------------------ Vielbein components
\DeclareDocumentCommand{\vi}{ t. t, t- s s m m}{
  \RenewDocumentCommand\MyAc{ m }{##1}
  \IfBooleanT{#1}{\RenewDocumentCommand\MyAc{ m }{ \mathring{##1} } }
  \IfBooleanT{#2}{\RenewDocumentCommand\MyAc{ m }{ \tilde{##1} } }
  \IfBooleanT{#3}{\RenewDocumentCommand\MyAc{ m }{ \bar{##1} } }
  \IfBooleanTF{#4}
  { \IfBooleanTF{#5} { \hat{\MyAc{e}}^{\hat{#6}}_{\hat{#7}} }{ \hat{\MyAc{e}}^{#6}_{{#7}} } }
  { \MyAc{e}^{{#6}}_{{#7}} } }
%%------------------------------ Compatibility
\newcommand\VI[2]{\hat{e}^{\hat{#1}}_{\hat{#2}}}
\newcommand\VIF[1]{\hat{\boldsymbol{e}}^{\hat{#1}}}
\newcommand\VIN[2]{\hat{E}^{\hat{#1}}_{\hat{#2}}}
\newcommand\VINF[1]{\hat{\boldsymbol{E}}_{\hat{#1}}}
\newcommand\hvi[2]{\hat{e}^{{#1}}_{{#2}}}
\newcommand\hvin[2]{\hat{E}^{{#1}}_{{#2}}}
\newcommand\hvif[1]{\hat{\boldsymbol{e}}^{{#1}}}
\newcommand\hvinf[1]{\hat{\boldsymbol{E}}_{{#1}}}
%% \newcommand\vi[2]{e^{{#1}}_{{#2}}}
\newcommand\vin[2]{E^{{#1}}_{{#2}}}
%% \newcommand\vif[1]{\boldsymbol{e}^{{#1}}}
\newcommand\vinf[1]{\boldsymbol{E}_{{#1}}}
%% \newcommand\Vi[2]{e^{\hat{#1}}_{\hat{#2}}}
%% \newcommand\Vin[2]{E^{\hat{#1}}_{\hat{#2}}}


%%---------------------------------------------
%%--------- Connections
%%---------------------------------------------
%%------------------------------ Christoffel symbols
\DeclareDocumentCommand{\ct}{ t. t, t- s s m m m }{
  \RenewDocumentCommand\MyAc{ m }{##1}
  \IfBooleanT{#1}{\RenewDocumentCommand\MyAc{ m }{ \mathring{##1} } }
  \IfBooleanT{#2}{\RenewDocumentCommand\MyAc{ m }{ \tilde{##1} } }
  \IfBooleanT{#3}{\RenewDocumentCommand\MyAc{ m }{ \bar{##1} } }
  \IfBooleanTF{#4}
  { \IfBooleanTF{#5} { \hat{\MyAc{\Gamma}}_{{#6}}{}^{\hat{#7}}{}_{\hat{#8}} }{ \hat{\MyAc{\Gamma}}_{{#6}}{}^{{#7}}{}_{{#8}} } }
  { \MyAc{\Gamma}_{{#6}}{}^{{#7}}{}_{{#8}} } }
%%------------------------------ Spin form
\DeclareDocumentCommand{\spif}{ t. t, t- s s m m }{
  \RenewDocumentCommand\MyAc{ m }{##1}
  \IfBooleanT{#1}{\RenewDocumentCommand\MyAc{ m }{ \mathring{##1} } }
  \IfBooleanT{#2}{\RenewDocumentCommand\MyAc{ m }{ \tilde{##1} } }
  \IfBooleanT{#3}{\RenewDocumentCommand\MyAc{ m }{ \bar{##1} } }
  \IfBooleanTF{#4}
  { \IfBooleanTF{#5} { \hat{\MyAc{\boldsymbol{\omega}}}^{\hat{#6}}{}_{\hat{#7}} }{ \hat{\MyAc{\boldsymbol{\omega}}}^{{#6}}{}_{{#7}} } }
  { \MyAc{\boldsymbol{\omega}}^{{#6}}{}_{{#7}} } }
%%------------------------------ Spin components
\DeclareDocumentCommand{\spi}{ t. t, t- s s m m m }{
  \RenewDocumentCommand\MyAc{ m }{##1}
  \IfBooleanT{#1}{\RenewDocumentCommand\MyAc{ m }{ \mathring{##1} } }
  \IfBooleanT{#2}{\RenewDocumentCommand\MyAc{ m }{ \tilde{##1} } }
  \IfBooleanT{#3}{\RenewDocumentCommand\MyAc{ m }{ \bar{##1} } }
  \IfBooleanTF{#4}
  { \IfBooleanTF{#5} { \hat{\MyAc{{\omega}}}_{\hat{#6}}{}^{\hat{#7}}{}_{\hat{#8}} }{ \hat{\MyAc{{\omega}}}_{{#6}}{}^{{#7}}{}_{{#8}} } }
  { \MyAc{{\omega}}_{{#6}}{}^{{#7}}{}_{{#8}} } }
%%------------------------------ Connections Compatibility
\newcommand{\conn}[3]{\left(\Gamma_{#1}\right)^{#2}{}_{#3}}
\newcommand{\connn}[3]{\Gamma_{#1}{}^{#2}{}_{#3}}
\newcommand{\Connn}[3]{\hat{\Gamma}_{#1}{}^{#2}{}_{#3}}
\newcommand{\CONNN}[3]{\hat{\Gamma}_{\hat{#1}}{}^{\hat{#2}}{}_{\hat{#3}}}
\newcommand{\hcon}[3]{\hat{\Gamma}_{#1}{}^{#2}{}_{#3}}
\newcommand\SPI[1]{\hat{\omega}_{\hat{#1}}}
\newcommand\SPIF[2]{\hat{\boldsymbol{\omega}}^{\hat{#1}}{}_{\hat{#2}}}
%% \newcommand\Spi[1]{\omega_{\hat{#1}}}
\newcommand\spc[1]{\omega_{{#1}}}
\newcommand\tspi[1]{\tilde{\omega}_{{#1}}}
%%\newcommand\spif[2]{{\boldsymbol{\omega}}^{{#1}}{}_{{#2}}}
\newcommand{\spife}[2]{\mathring{{\boldsymbol{\omega}}}^{#1}{}_{#2}}
\newcommand\tspif[2]{{\tilde{\boldsymbol{\omega}}}^{{#1}}{}_{{#2}}}
\newcommand\hspi[1]{\hat{\omega}_{{#1}}}
\newcommand\hspif[2]{\hat{\boldsymbol{\omega}}^{{#1}}{}_{{#2}}}
\newcommand\hspife[2]{\hat{\mathring{\boldsymbol{\omega}}}^{{#1}}{}_{{#2}}}

%%---------------------------------------------
%%--------- Curvature definitions
%%---------------------------------------------
%%------------------------------ Curvature form
\DeclareDocumentCommand{\rif}{ t. t, t- s s m m }{
  \RenewDocumentCommand\MyAc{ m }{##1}
  \IfBooleanT{#1}{\RenewDocumentCommand\MyAc{ m }{ \mathring{##1} } }
  \IfBooleanT{#2}{\RenewDocumentCommand\MyAc{ m }{ \tilde{##1} } }
  \IfBooleanT{#3}{\RenewDocumentCommand\MyAc{ m }{ \bar{##1} } }
  \IfBooleanTF{#4}
  { \IfBooleanTF{#5} { \hat{\MyAc{\bm{\mathcal{R}}}}{}^{\hat{#6}}{}_{\hat{#7}} }{ \hat{\MyAc{\bm{\mathcal{R}}}}{}^{{#6}}{}_{{#7}} } }
  { \MyAc{\bm{\mathcal{R}}}{}^{{#6}}{}_{{#7}} } }
%%------------------------------ Curvature components
\DeclareDocumentCommand{\ri}{ t. t, t- s s m m m }{
  \RenewDocumentCommand\MyAc{ m }{##1}
  \IfBooleanT{#1}{\RenewDocumentCommand\MyAc{ m }{ \mathring{##1} } }
  \IfBooleanT{#2}{\RenewDocumentCommand\MyAc{ m }{ \tilde{##1} } }
  \IfBooleanT{#3}{\RenewDocumentCommand\MyAc{ m }{ \bar{##1} } }
  \IfBooleanTF{#4}
  { \IfBooleanTF{#5} { \hat{\MyAc{\mathcal{R}}}_{{#6}}{}^{\hat{#7}}{}_{\hat{#8}} }{ \hat{\MyAc{\mathcal{R}}}_{{#6}}{}^{{#7}}{}_{{#8}} } }
  { \MyAc{\mathcal{R}}_{{#6}}{}^{{#7}}{}_{{#8}} } }
%%------------------------------ Curvature Compatibility
%%%%%%%%% Beware of the inconsistency between
%%%%%%%%% \Rif and \RIF
\newcommand{\RIF}[2]{\hat{\bm{\mathcal{R}}}^{\hat{#1}}{}_{\hat{#2}}}
\newcommand{\hRif}[2]{\hat{\bm{\mathcal{R}}}^{{#1}}{}_{{#2}}}
\newcommand{\Rif}[2]{\bm{\mathcal{R}}^{{#1}}{}_{{#2}}}
\newcommand{\tRif}[2]{\tilde{\bm{\mathcal{R}}}^{{#1}}{}_{{#2}}}

%%---------------------------------------------
%%--------- Contorsion definitions
%%---------------------------------------------
%%------------------------------ Contorsion form
\DeclareDocumentCommand{\kf}{ t. t, t- s s m m }{
  \RenewDocumentCommand\MyAc{ m }{##1}
  \IfBooleanT{#1}{\RenewDocumentCommand\MyAc{ m }{ \mathring{##1} } }
  \IfBooleanT{#2}{\RenewDocumentCommand\MyAc{ m }{ \tilde{##1} } }
  \IfBooleanT{#3}{\RenewDocumentCommand\MyAc{ m }{ \bar{##1} } }
  \IfBooleanTF{#4}
  { \IfBooleanTF{#5} { \hat{\MyAc{\bm{\mathcal{K}}}}^{\hat{#6}}{}_{\hat{#7}} }{ \hat{\MyAc{\bm{\mathcal{K}}}}^{{#6}}{}_{{#7}} } }
  { \MyAc{\bm{\mathcal{K}}}^{{#6}}{}_{{#7}} } }
%%------------------------------ Contorsion components
\DeclareDocumentCommand{\ko}{ t. t, t- s s m m m }{
  \RenewDocumentCommand\MyAc{ m }{##1}
  \IfBooleanT{#1}{\RenewDocumentCommand\MyAc{ m }{ \mathring{##1} } }
  \IfBooleanT{#2}{\RenewDocumentCommand\MyAc{ m }{ \tilde{##1} } }
  \IfBooleanT{#3}{\RenewDocumentCommand\MyAc{ m }{ \bar{##1} } }
  \IfBooleanTF{#4}
  { \IfBooleanTF{#5} { \hat{\MyAc{\mathcal{K}}}_{{#6}}{}^{\hat{#7}}{}_{\hat{#8}} }{ \hat{\MyAc{\mathcal{K}}}_{{#6}}{}^{{#7}}{}_{{#8}} } }
  { \MyAc{\mathcal{K}}_{{#6}}{}^{{#7}}{}_{{#8}} } }
%%------------------------------ Contorsion Compatibility
\newcommand{\Kf}[2]{{\bm{\mathcal{K}}}^{#1}{}_{#2}}
\newcommand{\cont}[3]{\mathcal{K}_{#1}{}^{#2}{}_{#3}}
\newcommand{\contf}[2]{\bm{\mathcal{K}}^{#1}{}_{#2}}
\newcommand{\hcont}[3]{\hat{\mathcal{K}}_{#1}{}^{#2}{}_{#3}}
\newcommand{\hcontf}[2]{\hat{\bm{\mathcal{K}}}^{#1}{}_{#2}}
\newcommand{\CONT}[3]{\hat{\mathcal{K}}_{\hat{#1}}{}^{\hat{#2}}{}_{\hat{#3}}}
\newcommand{\CONTU}[3]{\hat{\mathcal{K}}_{\hat{#1}}{}^{\hat{#2}\hat{#3}}}
\newcommand{\Contu}[3]{\hat{\mathcal{K}}_{{#1}}{}^{{#2}{#3}}}
\newcommand{\contu}[3]{{\mathcal{K}}_{{#1}}{}^{{#2}{#3}}}
\newcommand{\CONTF}[2]{\hat{\bm{\mathcal{K}}}^{\hat{#1}}{}_{\hat{#2}}}

%%---------------------------------------------
%%--------- Torsion definitions
%%---------------------------------------------
%%------------------------------ Torsion form
\DeclareDocumentCommand{\tf}{ t. t, t- s s m }{
  \RenewDocumentCommand\MyAc{ m }{##1}
  \IfBooleanT{#1}{\RenewDocumentCommand\MyAc{ m }{ \mathring{##1} } }
  \IfBooleanT{#2}{\RenewDocumentCommand\MyAc{ m }{ \tilde{##1} } }
  \IfBooleanT{#3}{\RenewDocumentCommand\MyAc{ m }{ \bar{##1} } }
  \IfBooleanTF{#4}
  { \IfBooleanTF{#5} { \hat{\MyAc{\bm{\mathcal{T}}}}^{\hat{#6}} }{ \hat{\MyAc{\bm{\mathcal{T}}}}^{{#6}} } }
  { \MyAc{\bm{\mathcal{T}}}^{{#6}} } }
%%------------------------------ Torsion Compatibility
\newcommand{\tor}{\mathcal{T}}
\newcommand{\tors}[3]{\mathcal{T}{}_{#1}{}^{#2}{}_{#3}}
\newcommand{\Tors}[3]{\hat{\mathcal{T}}{}_{#1}{}^{#2}{}_{#3}}
\newcommand{\TORS}[3]{\hat{\mathcal{T}}{}_{\hat{#1}}{}^{\hat{#2}}{}_{\hat{#3}}}
\newcommand{\torss}[3]{T_{#1}{}^{#2}{}_{#3}}
\newcommand{\Tf}[1]{\bm{\mathcal{T}}^{#1}}
\newcommand{\hTf}[1]{\hat{\bm{\mathcal{T}}}^{#1}}
\newcommand{\TF}[1]{\hat{\bm{\mathcal{T}}}^{\hat{#1}}}
\newcommand{\Tor}[2]{\mathcal{T}^{#1}{}_{#2}}

\newcommand\GAM[1]{{\gamma}^{\hat{#1}}}
%% \newcommand\Gam[1]{\gamma^{\hat{#1}}} 
\newcommand\gam[1]{\gamma^{{#1}}}
\newcommand\hgam[1]{\hat{\gamma}^{{#1}}}
%% %%%%%%%%%%%%%%%%%%
%% Problematic commands
%% with arXiv protocols
%% %%%%%%%%%%%%%%%%%%
\newcommand\NAB[1]{\hat{\nabla}_{\hat{#1}}}
\newcommand\Nab[1]{\nabla_{\hat{#1}}}
\newcommand\nab[1]{\nabla_{{#1}}}
\newcommand\PA[1]{\partial_{\hat{#1}}}
\newcommand\pa[1]{\partial_{{#1}}}
\newcommand\PAU[1]{\partial^{\hat{#1}}}
\newcommand\pau[1]{\partial^{{#1}}}
%% until here--- Change them before
%% trying to upload to arXiv
\newcommand\lf[1]{{\omega}^{{#1}}}
\newcommand\lft[1]{\hat{\omega}^{{#1}}}

\newcommand{\free}[1]{\mathring{#1}}
%% \newcommand{\ket}[1]{\left.\left|#1\right.\right>}
%% \newcommand{\bra}[1]{\left.\left<#1\right.\right|}
\renewcommand\bra[1]{\Bra{#1}}
\renewcommand\ket[1]{\Ket{#1}}
\newcommand{\bkt}[3]{\Braket{ {#1} | {#2} | {#3} } }
\newcommand{\bk}[2]{\Braket{ {#1} | {#2} } }
\newcommand{\comm}[2]{\left[#1,#2\right]}
\newcommand{\acomm}[2]{\left\{#1,#2\right\}}
\newcommand{\vev}[1]{\ensuremath{\left<#1\right>}}
\renewcommand{\set}[1]{\ensuremath{\Set{ #1 }}}

\newcommand{\relphantom}[1]{\mathrel{\phantom{#1}}}

\newcommand{\Riem}{\operatorname{Riem}}
\newcommand{\Ric}{\operatorname{Ric}}
\newcommand*{\diag}{\operatorname{diag}}
\newcommand{\id}{\operatorname{id}}
\newcommand{\tr}{\operatorname{tr}}
\newcommand{\Tr}{\operatorname{Tr}}
\newcommand{\Ker}{\operatorname{Ker}}
\renewcommand{\Im}{\operatorname{Im}}
\newcommand{\sgn}{\operatorname{sgn}}
\newcommand{\Ln}{\operatorname{Ln}}
\newcommand{\Ei}{\operatorname{Ei}}
\newcommand{\csch}{\operatorname{csch}}
\newcommand{\arcsinh}{\operatorname{arcsinh}}
\DeclareMathOperator\Br{Br}

\newcommand{\beq}{\begin{equation}}
\newcommand{\eeq}{\end{equation}}
\newcommand{\ber}{\begin{eqnarray}}
\newcommand{\eer}{\end{eqnarray}}

%% \renewcommand{\(}{\left(}
%% \renewcommand{\)}{\right)}
%% \renewcommand{\[}{\left[}
%% \renewcommand{\]}{\right]}

\newcommand{\uf}[2][]{\ensuremath{u_{#1}\(\vec{#2}\)}}
\newcommand{\ufb}[2][]{\ensuremath{\bar{u}_{#1}\(\vec{#2}\)}}
\newcommand{\vf}[2][]{\ensuremath{v_{#1}\(\vec{#2}\)}}
\newcommand{\vfb}[2][]{\ensuremath{\bar{v}_{#1}\(\vec{#2}\)}}
\newcommand\pol[2][]{\ensuremath{\varepsilon_{#1}(\vec{#2})}}
\newcommand\polc[2][]{\ensuremath{\varepsilon^*_{#1}(\vec{#2})}}
\newcommand{\ann}[3]{\ensuremath{#1\(\vec{#2},#3\)}}
\newcommand{\cre}[3]{\ensuremath{#1^\dag\(\vec{#2},#3\)}}
%% \newcommand{\uf}[2]{\ensuremath{u\(\vec{#1},#2\)}}
%% \newcommand{\ufb}[2]{\ensuremath{\bar{u}\(\vec{#1},#2\)}}
%% \newcommand{\vf}[2]{\ensuremath{v\(\vec{#1},#2\)}}
%% \newcommand{\vfb}[2]{\ensuremath{\bar{v}\(\vec{#1},#2\)}}
%% \newcommand{\ann}[3]{\ensuremath{#1\(\vec{#2},#3\)}}
%% \newcommand{\cre}[3]{\ensuremath{#1^\dag\(\vec{#2},#3\)}}

\newcommand{\dif}{{\mathrm{d}}}
\newcommand{\difn}[1]{{\mathrm{d}}^{#1}}
\newcommand{\dn}[2]{{\mathrm{d}}^{#1}\!{#2}\;}
\newcommand{\dbarn}[2]{\dbar{#1}{#2}\;}
\newcommand*{\de}[1]{\mathop{\mathrm{d}#1}\nolimits}% differential, facultative argoment between square parentheses
\newcommand*{\desec}[1][]{\mathop{\mathrm{d^2}#1}\nolimits}% second differential, facultative argoment between square parentheses
\newcommand{\der}[2]{\frac{\de{#1}}{\de{#2}}}% first derivative 
%\newcommand{\pder}[2]{\frac{\pa{}#1}{\pa{}{#2}}}% first derivative 
\newcommand{\inlineder}[2]{\mathrm{d}{#1}/\mathrm{d}{#2}}% in-line first derivative
\newcommand{\dersec}[2]{\frac{{\desec[#1]}}{\de[{#2}^2]}}% second derivative
%% \newcommand{\dx}{\de[x]}% frequently used differentials
%% \newcommand{\dy}{\de[y]}
%\newcommand{\df}{\de[f]}

%------------------
%--------- Format
%------------------
\newcommand{\out}[1]{{\color{red} {#1}}}
\newcommand{\pro}[1]{{\color{blue!70!black} {#1}}}
\newcommand{\hl}[1]{{\color{red} \bfseries{#1}}}

\newcommand\UTFSM{Departamento de F\'isica, Universidad T\'{e}cnica Federico Santa Mar\'\i a, \\ Casilla 110-V, Valpara\'iso, Chile}
\newcommand\UTFSMmat{Departamento de Matem\'aticas, Universidad T\'{e}cnica Federico Santa Mar\'\i a, \\ Casilla 110-V, Valpara\'iso, Chile}
\newcommand\CCTVal{Centro Cient\'ifico Tecnol\'ogico de Valpara\'iso, \\ Casilla 110-V, Valpara\'\i so, Chile}
\newcommand\CFF{Centro de F\'isica Fundamental,  Universidad de los Andes,\\ 5101 M\'erida, Venezuela}
\newcommand\PUCV{Instituto de F\'isica, Pontificia Universidad Cat\'olica de Valpara\'iso,\\ Casilla 4059, Valpara\'iso, Chile}
\newcommand\CECS{Centro de Estudios Cient\'ificos (CECs), Casilla 1469, Valdivia, Chile}

