\documentclass[aps,prd,12pt,superscriptaddress,showpacs,showkeys,longbibliography,reprint]{revtex4-1}
\usepackage{calrsfs} 
\usepackage{amsmath,amsthm,latexsym,amssymb,amsfonts}
\usepackage{xcolor}
\usepackage{textcomp}
\usepackage[%
  colorlinks=true,
  urlcolor=blue,
  linkcolor=blue,
  citecolor=blue
]{hyperref}
\usepackage{etoolbox}
\usepackage{breqn}

%% \makeatletter
%% \let\cat@comma@active\@empty
%% \makeatother

%------------------
%--------- Definitions
%------------------
\input{Def-article.tex}


\hypersetup{%
  pdftitle={Dimensional reduction on $S^1$ of Lanczos--Lovelock and Lovelock--Cartan theories of gravity},
  pdfauthor={Oscar Castillo-Felisola,}{Cristobal Corral,}{Simon del Pino,}{Francisca Ramirez.},
  pdfkeywords={Torsion,} {Generalised Gravity.},
  pdflang={English}
}


%------------------
%--------- Document
%------------------
\begin{document}

\title{Five Dimensional Kaluza--Klein Cosmology in Lovelock--Cartan Theory}

\author{Oscar \surname{Castillo-Felisola}}
\email{o.castillo.felisola@gmail.com}
\affiliation{\UTFSM.}
\affiliation{\CCTVal.}

\author{Cristobal \surname{Corral}}
\affiliation{\UTFSM.}
\affiliation{\CCTVal.}

\author{Sim\'on \surname{del~Pino}}
\email{simon.delpino.m@mail.pucv.cl}
\affiliation{\PUCV.}

\author{Francisca \surname{Ram\'irez}.}
\affiliation{\UTFSM.}

%% --------- Abstract
\begin{abstract}
 We study the Kaluza-Klein dimensional reduction of an $N$ dimensional Riemann--Cartan manifold when the compact section is $S^1$. First, we focus on the details of dimensional reduction in the presence of torsion to then study the particular Einstein--Cartan theory with the Gauss--Bonnet term in five dimensions. We look for the general cosmological solution of the Friedman-Robertson-Walker familly in the four-manifold to interpret the remaining fields. The model, treated in five dimensional vacuum, presents torsion both in the reduced manifold as well as in the compact section, manifest as a non-vanishing four dimensional matter field that adds to the effective energy density and pressure. This allows the model to present expanding periods as well as contracting universes, and a radius of the compact section that it asymtotes zero in some solutions. 
\end{abstract}

%% \pacs{02.40.Ma,04.50.Kd,04.90.+e}
%% \keywords{Affine Gravity, Torsion, Generalised Gravity.}


\maketitle

\section{Introduction}
The geometric theory of gravity, General Relativity (GR), has been highly successful in describing the physics at large scales. Two major predictions were recently confirmed in one experiment that invoque two of the most fascinating conepts that this theory developed: the propagation of ripples of spacetime, gravitational waves, originated by the merge of two black holes where detected by the LIGO collaboration \cite{Abbott:2016blz} confirming the existence of both, astrophysical balck holes as well as the unprecedented nature of the signal detected. With this, we can rely in the geometric nature of this interaction. At the galactic scale, the gravitational lensing produced by the local distribution of energy and matter suggest the presence of a kind of matter that is not seen through telescopes. This abundance of matter in the galaxy is also compatible with the velocity profyle of stars at the outer regions of it and is needed to reach the current state of structures formation. This matter would interact mostly, if not only, through gravity \cite{Sofue:2000jx}. At the cosmic scale, observatinal evidence points out an accelerated current state of the Universe \cite{Riess:1998cb}. Such a behaviour needs the existence of a form of vacuum energy of exotic nature to give account of the accelerated expansion. This is evidence of unknown physics, and it has been the motivation for looking for new gravitational degrees of freedom beyond Einsten's GR. Among the familly of extended gravity theories, higher dimensional gravity can give us a clue to explore the manifestation of new degrees of freedom as material fields of certain class. Along this lines is that we consider here an extended model of gravity given by the so called Lovelock action~\cite{Lovelock:1971yv}. 
The Einstein--Hilbert action with cosmological constant is the unique action in four dimensions that yields second order field equations for the metric. In higher dimensions, however, this does not remains true. A particular combination of squared terms in the curvature can be added to the gravitational action to give rise a second order system for the metric with no ghosts. The particular quadratic term is called Gauss--Bonnet term, and in $N$ dimensions reads
\begin{equation}\label{GB}
  \mathcal{L}_{GB}=\text{d}^N \!x \,\sqrt{-g}\left(\tilde{R}^2-4\tilde{R}_{\mu\nu}\tilde{R}^{\mu\nu}+\tilde{R}_{\alpha\beta\mu\nu}
  \tilde{R}^{\alpha\beta\mu\nu}\right),
\end{equation}
where $\tilde{R}_{\alpha\beta\mu\nu}$ is the Riemannian curvature of a manifold with metric $g_{\mu\nu}$ and $g$ its determinant. $\tilde{R}_{\mu\nu}$ and $\tilde{R}$ are the Ricci tensor and Ricci scalar built with the curvature.

Written in four dimensions, Eq.~\eqref{GB} adds no dynamics to the metric, since it represents a topological invariant proportional to the Euler characteristic and therefore can always be written locally as a boundary term, while in lower dimensions vanishes identically. Is then natural to extend gravitational Lagrangians by dimensional continuation of Euler densities to build a general theory for a metric in arbitrary dimensions, while keeping the two important features of Einstein's General Relativity just mentioned: the absence of ghosts and second order field equations. These family of Lagrangians for gravity where introduced by Lovelock and bear his name. It can be built in arbitrary dimensions as a linear combination of dimensionally continued topological invariants. Equivalently, the previous Lagrangian can be written in exterior forms, that provide shorter expressions together with a natural framework for further generalizations, in $N$ dimensions reads
\begin{equation}\label{GB2}
\mathcal{L}_{GB} = \epsilon_{a_1...a_N} \tilde{R}^{a_1a_2}\wedge\tilde{R}^{a_3a_4}\wedge e^{a_5}\wedge...\wedge e^{a_N}.
\end{equation}
This Lagrangian was also identified as the low energy correction for a spin two field in string theory~\cite{Zwiebach:1985uq}. Thus, the consideration of further Lovelock terms in a higher dimensional manifold can be interpreted as semiclassical corrections of a gravity action.
The higher dimensional scenario seems to be a natural framework for unification. Early in the last century, it was realized by T. Kaluza and developed by O. Klein, that the existence of one extra direction in Einstein geometric theory of gravity would give rise a unified picture of gravity and electromagnetism together with a spectrum of new heavy particles~\cite{Kaluza:1921tu,*Klein:1926tv}. This opened the posibility of a geometrical understanding of interactions given by non-Abelian gauge groups as a concequence of the topology of a spatial compact section in a higher dimensional spacetime. Eventhough the original model, and further developments have not yet reached a realistic phenomenology, it is clear that this possibility of unification of internal and spacetimes symmetries, together with a material manifestation of geometric degrees of freedom provides a fruitful arena for looking for clues of unity among the interactions. The compactification of higher Lovelock terms has been considered \cite{MuellerHoissen:1985mm,MuellerHoissen:1989yv} and its consequences in cosmology in \cite{MuellerHoissen:1985ij,Deruelle:1986iv,Deruelle:1989fj}. In the Riemannian five dimensional theory, exact wormholes solutions have been found in vacuum \cite{Dotti:2006cp,Dotti:2007az}, and suported by non-exotic matter, making the sign of the Gauss-Bonnet coupling the source of exotic behaviour \cite{Mehdizadeh:2015jra}, higher order Lovelock models \cite{Mehdizadeh:2015dta}, as well as in higher order compactified Lovelock theories with torsion \cite{Canfora:2008ka} also have these solutions. Classically, wormholes must be supported by a kind of phantom energy, compatible with the cosmological hypothesis. We see their theoretical existence in these extended models as a good omen for the search of new degrees of freedom that could provide explanation for the exotic matter-energy abundance in the Universe. \\
Torsional degrees of freedom, on the other hand, as a further feature of spacetime had a \textit{renaissance} with the works of D. W. Sciama and T. Kibble \cite{Kibble:1961ba} in Poincar\'e gauge theory. In that framework, spacetime torsion arises naturally as the field strength of a gauge connection for the Poincar\'e group (see \cite{Hehl:1976kj,Blagojevic:2002du} for a complete review). The pose of gravity as a gauge theory has major benefits in the problem of a unifyied picture for interactions since it allows it to interpret it as arising by means of a localization procedure, where spacetime is a Riemann-Cartan manifold. In this context, it has been shown that torsional functions produce an accelerating profyle compatible with the current state of the Universe \cite{Shie:2008ms}, as well as non-minimal coupling of Euler forms in Riemann--Cartan spacetimes allow torsion to behave as matter of exotic characteristics \cite{Toloza:2013wi} customary for expansion.
It becomes essential to understand Kaluza--Klein's ideas in the presence of torsion. Work on the subject reports the phenomenology of extra dimensional torsion \cite{Kalinowski:1980da}, metric dependent torsional models of extra dimensions \cite{Shankar:2012vd} and its consequences in cosmology \cite{Chen:2009ep}, compatification of higher dimensional Brans--Dicke models with torsion \cite{German:1993bq}. Torsion free solutions where found in \cite{Aros:2007nn} for models in first order compactified gravity.\\
Eventhough at the macroscopic scale, spacetime seems to lack torsion, there is no logical constraint to assume it non-existing, as E. Cartan argued, moreover since vacuum predictions of Einstein--Cartan theory hold the experimental tests of General Relativity.\\
This work is organized as follows: in Sec.~\ref{KK} we discuss the introduction of torsional degrees of freedom compatible with the Kaluza--Klein geometric reasoning, as well as fix our notation and conventions. Sec.~\ref{5EGB} studies the dynamics of the general Lovelock--Cartan action in five-dimensional spacetime and its dimensional reduction. Then, in Sec.~\ref{cosmos} we look for cosmological solutions in the reduced four-dimensional manifold of the Friedmann--Robertson--Walker class, with torsion both in the reduced manifold and as new dynamical fields arising from the compact section taken to be $S^1$. Conslusions and remarks are given next in Sec.~\ref{conclusions}. We also found appropriate to incorporate a number of appendices to make this work a self contained article.
%_-----------------------------------------------------------------------------------------

\section{Kaluza--Klein Ansatz in Riemann--Cartan Spacetimes\label{KK}}
Let $M_N$ be a $N$ dimensional differential manifold. Every quantity defined on $M_N$ will wear hats $\hat{x}$. Capital Greek characters (coordinate indices) and capital Latin characters (Lorentz indices) run along the $N$ dimensions, while lower case run in the $(N-1)$ dimensional reduced manifold.\\
Let $\hat{\omega}^{AB}$ be the spin connection 1-form that describes the affine structure of the manifold and $\hat{e}^A=\hat{e}^{A}_{\ \Gamma}\,\text{d}\hat{x}^\Gamma$ be the vielbein 1-form that defines the metric structure of the same manifold through the relation $\hat{g}_{\Gamma\Delta}=\hat{e}^{A}_{\ \Gamma}\hat{e}^{B}_{\ \Delta}\hat{\eta}_{AB}$. The vielbein is the mapping that relates coordinate indices (Greek characters) with the Lorentz indices (Latin characters). In our convention the flat Lorentzian metric will be $\hat{\eta}_{AB}=\,$diag$\,(-,+,...,+)$, and the $N$ dimensional Levi-Civita symbol $\hat{\epsilon}_{A_1...A_{N-1}}$ is reduced by defining
\begin{align*}
\epsilon_{a_1...a_{N-1}}\equiv\hat{\epsilon}_{a_1...a_{N-1}N}.
\end{align*} 
Our convention fixes $\hat{\epsilon}_{01...N}=1$.\\
Riemannian fields (torsion free) will also be explicited by a tilde, as it was done in Eq.~\eqref{GB} and Eq.~\eqref{GB2} for the curvature.\\
Due to the topology of the manifold, we can expand the depenence of the fields on the extra coordinate, $z$, in Fourier series as  
\begin{align}\label{Fourier}
\varrho(\hat{x}^\Gamma)&=\sum_n\varrho_{(n)}(x^\gamma)e^{in z}.
\end{align}
For the rest of the paper, we will focus in the $n=0$ mode of the expasion, also refered to as the low energy sector.
\subsection{Metric and affine structure}
The Kaluza-Klein (KK) ansatz for the metric lies on the premise that the compact section of $M_N$ is orthogonal to the rest of the manifold at each point. This leads to the conclusion that the metric has the following structure
\begin{equation}
  \hat{g}_{\Gamma\Delta} =
  \begin{pmatrix}
    g_{\gamma\delta} +\frac{\hat{g}_{\gamma z}\hat{g}_{\delta z}}{\hat{g}_{zz}}&\hat{g}_{\gamma z}\\
    \hat{g}_{z\delta} & \hat{g}_{zz}
  \end{pmatrix}
  =
  \begin{pmatrix}
    g_{\gamma\delta} + \phi A_\gamma A_\delta&\phi A_\gamma\\
    \phi A_{\delta} & \phi
  \end{pmatrix},
\end{equation}
and its inverse
\begin{equation}
  \hat{g}^{\Gamma\Delta}=
  \begin{pmatrix}
    g^{\gamma\delta}&-A^\gamma\\
    -A^{\delta} & \phi^{-1}+A^2
  \end{pmatrix}.
\end{equation}
It introduces a scalar field $\phi$ and a vector field $A_\mu$ as new gravitational degrees of freedom materialized in the reduced manifold.
The vielbein that keeps this structure for the metric has the following form
\begin{equation}
  \label{Dvielbein}
  \hat{e}^A_{\ \ \Gamma} =
  \begin{pmatrix}
    \hat{e}^a_{\ \gamma}& 0\\
    \hat{e}^N_{\ \gamma} & \hat{e}^N_{\ \ z}
  \end{pmatrix}
  =
  \begin{pmatrix}
    e^a_{\ \gamma}& 0\\
    \sqrt{\phi}A_\gamma & \sqrt{\phi}
  \end{pmatrix},
\end{equation}
and its inverse, $\hat{E}_A^{\ \ \Gamma}$ defined such that $\hat{E}_A^{\ \ \Gamma}\hat{e}^A_{\ \ \Delta}=\delta^\Gamma_{\Delta}$, read
\begin{equation}
  \label{Dinversevielbein}
  \hat{E}_A^{\ \ \Gamma} =
  \begin{pmatrix}
    \hat{E}_a^{\ \gamma}& 0\\
    \hat{E}_a^{\ z} & \hat{E}_N^{\ \ z}
  \end{pmatrix}
  =
  \begin{pmatrix}
    E_a^{\ \gamma}& 0\\
    -A_a & \sqrt{\phi}^{-1}
  \end{pmatrix},
\end{equation}
and it shows that the compact section $S^1$ has its own tangent space $T_pS^1$ at each point $p\in M_N$, and is independent to $T_pM_{N-1}$ in the sense that they do not mix. It seems reasonable to think that if we want two manifolds to be orthogonal to one another, they will not mix vectors after the tangent mapping.\\
The basis of vielbeins is defined modulo Lorentz transformations. Any transformed basis 
$
\hat{e}^{A\prime}=\hat{\Lambda}^A_{\ B}\hat{e}^B,
$
where $\hat{\Lambda}$ is a Lorentz matrix, is as suitable as $\hat{e}^A$, and therefore share the structure in Eq.~\eqref{Dvielbein}. The Lorentz transformations are then reduced by this fact, imposing over the Loretnz matrix 
$\hat{\Lambda}^a_{\ N}=0.
$

%_-----------------------------------------------------------------------------------------


A Riemannian connection compatible with a $N$ dimensional vielbein, Eq.~\eqref{Dvielbein}, is built under the premise that $\mbox{d}\hat{e}^A + \hat{\tilde{\omega}}^{AB} \wedge \hat{e}^A = 0$. Thus, we find
\begin{align}
  \label{DRiemannconnection}
  \hat{\tilde{\omega}}^{ab}&=\tilde{\omega}^{ab}-\frac{1}{2}\sqrt{\phi}F^{ab}\hat{e}^N,\\
  \hat{\tilde{\omega}}^{Na}&=\frac{1}{2}\sqrt{\phi}F^a_{\ \ l}e^l+\frac{1}{2}\partial^a\ln\phi\hat{e}^N,
\end{align}
where $\tilde{\omega}^{ab}$ is the Riemannian spin connection of the reduced manifold. The construction of a theory for a spacetime with spin connection \eqref{DRiemannconnection} leads to the original KK theory. It gives rise the usual $U(1)$-invairant field strength, 
\begin{align*}
F=\text{d}A=\frac{1}{2}F_{ab}\, e^a\wedge e^b,
\end{align*} 
non-minimally coupled to the scalar field.\\
Nevertheless, $M_N$ is in this case, not entirely described by $\hat{\tilde{\omega}}^{AB}$, but by a more general spin connection, independent of the metric degrees of freedom (see Appendix \ref{Riemann-Cartan} for details). We can assume a general connection 1-form of the same type (regarding its $M_{N-1}\times S^1$ decomposition) as in eq.~\eqref{DRiemannconnection}. Thus
the most general spin connection on $M_N$ compatible with the KK decomposition is given by 
\begin{equation}\label{Nconnection}
  \hat{\omega}^{AB} \equiv
  \begin{pmatrix}
    \omega^{ab}+\alpha^{ab}\hat{e}^N&\beta^a+\gamma^a\hat{e}^N\\
    -\beta^b-\gamma^b\hat{e}^N&0
  \end{pmatrix}.
\end{equation}
It contemplates the torsional degrees of freedom appart of the usual metric ones. 
The decomposition \eqref{Nconnection} add new independent fields into the picture. Here $\alpha^{ab}$ is a $0$-form, $\beta^a$ a vector valued $1$-form and $\gamma^a$ a $0$-form that admit the Fourier expansion \eqref{Fourier} regarding their dependence on the extra dimension. Here, as we pointed out earlier, we will take the $0$-mode of such an expansion.

%_-----------------------------------------------------------------------------------------

\subsection{Curvature and torsion}
Using the definition of curvature
$\hat{R}^{AB}=\mbox{d}\hat{\omega}^{AB}+\hat{\omega}^A_{\ \ C}\wedge\hat{\omega}^{CB}$,
we find
\begin{align}\label{R1}
  \hat{R}^{ab}&=R^{ab}+\sqrt{\phi}\alpha^{ab}F-\beta^a\wedge\beta^b\notag\\
  &+\left(\mbox{D}\alpha^{ab}+\frac{1}{2}\alpha^{ab}\mbox{d}\ln\phi-2\beta^{[a}\gamma^{b]}\right)\wedge\hat{e}^N,\\
  \label{R2}
  \hat{R}^{aN}&=\left(\mbox{D}\beta^a+\sqrt{\phi}\gamma^a F\right)\notag\\
  &-\left(\alpha^a_{\ b}\beta^b-\mbox{D}\gamma^a-\frac{1}{2}\gamma^a\mbox{d}\ln\phi\right)\wedge\hat{e}^N.
\end{align}
where we have used the fact that
$
  \mbox{d}\hat{e}^N=\frac{1}{2}\mbox{d}\ln\phi\wedge\hat{e}^N+\sqrt{\phi}F.
$
Similarly from the definition of torsion
$\hat{T}^A = \hat{\mbox{D}}\hat{e}^A=\mbox{d}\hat{e}^A+\hat{\omega}^A_{\ \ B}\wedge\hat{e}^B$
we find its distinctive parts to be
\begin{align}\label{T1}
  \hat{T}^a &= T^a+\left(\beta^a-\alpha^a_{\ \ b}e^b\right)\wedge\hat{e}^N,\\
  \label{T2}
  \hat{T}^N &= \sqrt{\phi}F-\beta_b\wedge e^b+\left(\frac{1}{2}\mbox{d}\ln\phi+\gamma_be^b\right)\wedge\hat{e}^N.
\end{align}

\subsubsection{Bianchi identities}

Considering the Bianchi identities of the KK structure described above, we find relevant information about the new fields. Taking the exterior covariant derivative of the field strengths
$\hat{R}$ and $\hat{T}$, the Bianchi identities are
\begin{align*}
  \hat{\text{D}}\hat{R}^{AB} &= 0, & \hat{\text{D}}\hat{T}^A &= \hat{R}^A_{\ \ B}\wedge\hat{e}^B.
\end{align*}
A careful decomposition of the first one in its distinctive parts, gives the Bianchi identity for the curvature of the reduced spacetime together with the second derivative rules
\begin{align*}
  \text{D}R^{ab} &=0, & \text{D}\mbox{D}\alpha^{ab} &=R^a_{\ \ l}\alpha^{lb}+R^b_{\ \ l}\alpha^{al},\\
  \text{D}\text{D}\beta^a &= R^a_{\ \ b}\beta^{b}, & \text{D}\text{D}\gamma^a &= R^a_{\ \ b}\gamma^{b}.
\end{align*}
This is taken as proof of the tensorial condition of these new fields. While consideration of the second one gives nothing but its equivalent for the manifold $M_{N-1}$, this is
\begin{equation*}
  \mbox{D}T^a=R^a_{\ \ b}\wedge e^b.
\end{equation*}





%_-----------------------------------------------------------------------------------------

\section{Five Dimensional Einstein--Gauss--Bonnet Reduction\label{5EGB}}
We will work in this section, the dimensional reduction of a gravitational Lagrangian that includes quadratic terms in the curvature, namely the Gauss--Bonnet term. The action is given by
\begin{widetext}
  \begin{equation}\label{action5EGB}
    I=\int\limits_{M_5}\hat{\epsilon}_{ABCDE}\Big(\frac{\alpha_0}{5}\hat{e}^A\wedge\hat{e}^B\wedge\hat{e}^C\wedge
    \hat{e}^D\wedge\hat{e}^E
    +\frac{\alpha_1}{3}\hat{R}^{AB}\wedge\hat{e}^C\wedge\hat{e}^D\wedge\hat{e}^E
    +\alpha_2\hat{R}^{AB}\wedge\hat{R}^{CD}
    \wedge\hat{e}^E\Big),
  \end{equation}
where $\alpha_0$, $\alpha_1$ and $\alpha_2$ are coupling constants. Variations with respect to the vielbein and spin connection give the equations
  \begin{align}\label{delta e}
    \hat{\epsilon}_{ABCDE}\Big(\alpha_0\hat{e}^A\wedge\hat{e}^B\wedge\hat{e}^C\wedge\hat{e}^D
    +\alpha_1\hat{R}^{AB}\wedge\hat{e}^C\wedge\hat{e}^D
    +\alpha_2\hat{R}^{AB}\wedge\hat{R}^{CD}\Big)&=0,
    \\\label{delta w}
    \hat{\epsilon}_{ABCDE}\Big(\alpha_1\hat{e}^C\wedge\hat{e}^D+
    2\alpha_2\hat{R}^{CD}\Big)\wedge\hat{T}^E&=0,
  \end{align}
  respectively. 
\end{widetext}
In this work, we will focus on the general case when
\begin{equation}\label{delta}
\Delta\equiv\alpha_1^2-4\alpha_0\alpha_2\neq 0.
\end{equation}
Otherwise, the Lagrangian becomes a gauge invariant Lagrangian for a larger group, namely the $AdS_5$ group, and is recognized as its Chern--Simons form \cite{Zanelli:2005sa}, \cite{Troncoso:1999pk}. In general, Lovelock--Cartan (LC) theory allows the torsional term
\begin{align*}
\mathcal{L}_{T}\sim\hat{R}_{AB}\wedge\hat{T}^A\wedge\hat{e}^B,
\end{align*}
that in five dimensions is a boundary term proportional to the Nieh-Yan invariant, thus it does not participate in the dynamics. Action \eqref{action5EGB} is the general LC theory in five dimensions.\\
In terms of the KK ansatz presented in previous sections, the equations of motion can be decomposed in to its distinctive parts. Eq.~\eqref{delta e} leads to the following set
\begin{align}\label{equation}
\epsilon_{abcd}\Big[\alpha_0 e^a\wedge e^b\wedge e^c+\frac{1}{2}\alpha_1\left(M^{ab}-L^a\wedge e^b\right)\wedge e^c&\notag\\
-\alpha_2\left(L^a\wedge M^{bc}+K^a\wedge N^{bc}\right)\Big]&=0,\\
\epsilon_{abcd}K^a\wedge\left(\alpha_1e^b\wedge e^c+2\alpha_2M^{bc}\right)&=0,\\
\epsilon_{abcd}\big(\alpha_0 e^a\wedge e^b\wedge e^c\wedge e^d&\notag\\
+\alpha_1M^{ab}\wedge\wedge e^c\wedge e^d+\alpha_2M^{ab}\wedge M^{cd}\big)&=0,\\
\epsilon_{abcd}\left(\alpha_1N^{ab}\wedge e^c\wedge e^d+2\alpha_2N^{ab}M^{cd}\right)&=0,
\end{align}
while Eq.~\eqref{delta w} gives
\begin{align}
\epsilon_{abcd}\Big[\alpha_1\left(e^c\wedge e^d\wedge Z-2e^c\wedge T^d\right)
+2\alpha_2\Big(N^{cd}\wedge W&\notag\\
+M^{cd}\wedge Z+2L^c\wedge T^d
+2K^c\wedge V^d\Big)\Big]&=0,\\
\epsilon_{abcd}\left[\alpha_1e^c\wedge e^ d\wedge W+2\alpha_2\left(M^{cd}\wedge W+2K^c\wedge T^d\right)\right]&=0,\\
\epsilon_{abcd}\left(\alpha_1e^b\wedge e^c+2\alpha_2M^{bc}\right)\wedge T^d&=0,\\
\epsilon_{abcd}\left[\alpha_1e^b\wedge e^c V^d+2\alpha_2\Big(M^{bc}\wedge V^d+N^{bc}\wedge T^d\Big)\right]&=0.
\end{align}
The fields $M^{ab}$, $N^{ab}$, $L^a$, $K^a$, $W$, $V^a$ and $Z$ are defined through \eqref{R1}, \eqref{R2}, \eqref{T1} and \eqref{T2} such that
\begin{align*}
\hat{R}^{ab}&=M^{ab}+N^{ab}\wedge\hat{e}^5, & \hat{R}^{5a}&=K^a+L^a\wedge\hat{e}^5,\\
\hat{T}^a&=T^a+V^a\wedge\hat{e}^5, & \hat{T}^5&=W+Z\wedge\hat{e}^5.
\end{align*}





%_-----------------------------------------------------------------------------------------
\section{Dimensionally Reduced EGB Cosmology}\label{cosmos}
\subsection{Cosmological ansatz}
We will look for cosmological solutions of the equations of motion. For that, we demand the symmetries assumed by the cosmological principle, namely, isotropy and homogeneity of the fields involve. This is achieved by making the Lie derivative in the direction of the Killing vectors of the symmetries to vanish for each field. Appendix \ref{homotropic} is devoted to the details of how to reach the general ansatz that we present next. The vielbein compatible with the Friedman-Robertson-Walker metric is determined by one function called scale factor $a(t)$,
\begin{align}
  \label{vielbein cosmo}
  e^0&=\mbox{d}t, & e^1&=\frac{a(t)}{\sqrt{1-kr^2}}\mbox{d}r,\\
  e^2&=a(t)r\mbox{d}\theta, & e^3&=a(t)r\sin\varphi\mbox{d}\varphi, 
\end{align}
where $k=+1,0,-1$ determines if the spatial section is closed, flat or an open $3$-space respectivelly. The scalar field is a time-dependent function
\begin{equation}
  \phi=\phi(t).
\end{equation}
On the other hand, the spin connection adds two functions we called $\omega(t)$ and $f(t)$,
\begin{align}
  \omega^{0i}&=\omega(t) e^i,\\
  \omega^{12}&=-\frac{\sqrt{1-kr^2}}{a(t)r}e^2-f(t)e^3,\\
  \omega^{13}&=-\frac{\sqrt{1-kr^2}}{a(t)r}e^3+f(t)e^2,\\
  \omega^{23}&=-\frac{\cot\theta}{a(t)r}e^3-f(t)e^1.
\end{align}
The non-vanishing components of the spin connection induced by the compact section of the manifold are
\begin{align}
  \beta^0=-b(t)e^0,&\ \beta^i=\beta(t)e^i,\\
  \label{gamma cosmo}
  \gamma^0&=-\gamma(t).
\end{align}
For this ansatz, the $1$-form $A$ has the form $A_t(t)\text{d}t$, which can be shown to have vanishing field strength $F$. Thus without loss of generality, it can be set to zero by means of a $U(1)$ gauge transformation, or equivalently, a diffeomorphism transformation along the $z$-direction.\\
The equations of motion for the cosmological ansatz gives a system of differential equations for the time dependent functions defined above. Curvature and torsion for this ansatz can be seen in Appendix \ref{homotropic}. We found that the only non-Riemannian branch demands  $\beta^a=0$. Thus $b(t)=\beta(t)=0$. Otherwise, the system degenerates at the CS point. The final set of equations is the following:
\begin{align}
  \label{eqn1}
  h\left[\alpha_1+2\alpha_2\left(\omega^2+\frac{k}{a^2}-f^2\right)\right]+4\alpha_2f\left(\dot{f}+Hf\right)&=0,\\
  2\alpha_0+\alpha_1\left(\dot{\omega}+2\omega^2+\frac{k}{a^2}-f^2\right)\notag\\
  +2\alpha_2\left(\dot{\omega}+\omega^2\right)\left(\omega^2+\frac{k}{a^2}-f^2\right)&=0,\\
  2\alpha_2\left(\frac{1}{2}\dot{\Phi}+\gamma\right)f\left(\dot{f}+Hf\right)
  +h^2\left(\alpha_1-2\alpha_2\omega\gamma\right)&=0,\\
  \omega\left(\frac{1}{2}\dot{\Phi}+\gamma\right)+\dot{\gamma}+\frac{1}{2}\gamma\dot{\Phi}-\frac{\alpha_1}{2\alpha_2}&=0,\\
  \left(\omega^2+\frac{k}{a^2}-f^2\right)\left(\alpha_1-2\alpha_2\omega\gamma\right)-\alpha_1\omega\gamma+2\alpha_0&=0,\\
  \label{eqn2}
  \left(\dot{\omega}+\omega^2\right)\left(\alpha_1-2\alpha_2\omega\gamma\right)-\alpha_1
  \omega\gamma+3\alpha_0-\frac{\alpha^2_1}{4\alpha_2}&=0,
\end{align}
where dotted functions stands for its time-derivatives. We have defined for simplicity the functions $h(t)=\omega(t)-H(t)$, where $H=\dot{a}/a$ is the Hubble function, and $\Phi=\ln\phi$.


\subsection{Solutions}

The system is integrated developing two non-Riemannian branches, one for each value of the parameter $u_\pm$ defenied to be
\begin{equation}\label{u}
  u_\pm=\frac{2\alpha_1\pm\sqrt{6\Delta}}{4\alpha_2},
\end{equation}
where $\Delta$ was given in Eq.~\eqref{delta}. Eqs.~\eqref{eqn1} to~\eqref{eqn2} reduced to the Riemannian system when \mbox{$f=h=0$} and \mbox{$\gamma=-\dot{\Phi}/2$}. The details of such a model can be seen in Refs.~\cite{Deruelle:1986iv,Deruelle:2003ck,Henriques:1986jw,Ishihara:1986if,Kleidis:1997mu}.

Due to Eq.~\eqref{u}, the solutions will be valid in the region of parameter space where $\Delta>0$. The function $\omega(t)$ satisfies the equation $\dot{\omega}+\omega^2+u_{\pm}=0$, that for the three significantly different values of $u_\pm$, has the following solutions
\begin{align}
  u_\pm>0&:\omega(t)=-\sqrt{u_\pm}\tan\left[\sqrt{u_\pm}\left(t-t_0\right)\right],\\
  u_\pm=0&:\omega(t)=\left(t-t_0\right)^{-1},\\
  u_\pm<0&:\omega(t)=\sqrt{-u_\pm}\tanh\left[\sqrt{-u_\pm}\left(t-t_0\right)\right],
\end{align}
where $t_0$ is an integration constant to be fixed. We relegate the time dependence of the remaining fields to these expressions for $\omega$ and list their explicit time dependence in Appendix~\ref{solutions t}. The solutions read
\begin{align}
  \gamma(\omega)&=\frac{u_\pm}{\omega},\\
  \phi(\omega)&=\frac{\phi_0\,\omega^2}{\omega^2+u_\pm}\exp\left[\frac{\mp\sqrt{6\Delta}}{4\alpha_2\left(\omega^2+u_\pm\right)}\right],\\
  a(\omega)&=\frac{a_0}{\sqrt{\omega^2+u_\pm}}\exp\left[\frac{\pm\sqrt{6\Delta}} {24\alpha_2\left(\omega^2+u_\pm\right)}\right],\\
  f^2(\omega)&=\left(\omega^2+u_\pm\right)\frac{k}{a_0^2}\exp\left[\frac{\mp\sqrt{6\Delta}}{12\alpha_2\left(\omega^2+u_\pm\right)}\right]\notag\\
  &+\omega^2-Y_\pm,
\end{align}
where we have defined the constant
\begin{equation*}
  Y_{\pm}=\mp2\frac{\alpha_1 u_\pm-2\alpha_0}{\sqrt{6\Delta}}=-\frac{3\alpha_1\pm\sqrt{6\Delta}}{6\alpha_2},
\end{equation*}
and where $a_0$ and $\phi_0$ are integration constants.

\subsubsection{Bouncing universes}
For $u_\pm>0$ the solutions are periodic. In that case, the scale factor starts from a singular point and reach a future one after a time $t_{\text{crunch}}=\pi/\sqrt{u_\pm}$. Depending on the particular sector of parameter space, the solution undergoes an expanding and contracting age that allow it to reach a minimum value without collapsing, to expand again before a big crunch. Otherwise, it reaches a maximum value before collapse. It also shows an inflationary phase at early times when the scale factor grows exponentially.


\subsubsection{Expanding and contracting universes}
For $u_\pm\leq 0$, depending on the sign of $\alpha_2$ and the $\pm$-sign in $u_\pm$, the scale factor will infinitely expand while the scalar field reach an asymtotic vanishing value for late cosmic times, after a period of expansion. This is interpreted as the late cosmic time behaviour of the radius of the compact spacial section $S^1$, that for this familly of solutions provides a natural explanation for its scale. Otherwise, in contrast to the previous behaviour, the scale factor contracts to an asymtotic vanishing value while the scalar field has no upper bound. In the case of strictly negative $u_\pm$, for either contracting or expanding scale factor, its initial value remains finite, thus the solutions do not present an original singularity. It has been reported that the GB term in a gravitational action can prevent the universe to expand from an initial singularity \cite{Deruelle:1986iv,Henriques:1986jw,Ishihara:1986if}, which is also the case here. 

\subsubsection{Effective energy density and pressure}
Eq.~\eqref{equation} take the familiar form
\begin{align}\label{einstein equation}
\frac{1}{2}\epsilon_{abcd}\left(\alpha_0e^a\wedge e^b\wedge e^c+\frac{1}{2}\alpha_1\tilde{R}^{ab}\wedge e^c\right)&=\tau^{eff}_d,
\end{align}
when we use \eqref{curvature decomp} and define an effective energy-momentum form as
\begin{align}
\tau_a^{eff}&=\frac{1}{2}\epsilon_{abcd}\bigg[\frac{1}{2}\alpha_1\left(\kappa^b_{\ l}\wedge\kappa^{lc}+\tilde{\text{D}}\kappa^{bc}\right)\wedge e^d\notag\\
&+\left(\text{D}\gamma^b+\frac{1}{2}\gamma^b\text{d}\ln\phi\right)\wedge\left(\frac{1}{2}\alpha_1e^c\wedge e^d+\alpha_2R^{cd}\right)\bigg].
\end{align}
The contributions of $F$, $\alpha^{ab}$ and $\beta^a$ have not been taken into account since for all practical purposes of this article, they vanish. The left hand side of \eqref{einstein equation} is Eintein's $3$-form, that in components gives Einstein's tensor. We identify the right hand side as an effective energy-momentum form that contemplates the torsional functions and the higher dimensional metric degrees of freedom as part of the gravitating matter of a four dimensional Universe in the form of a perfect fluid. For this energy-momentum form, we identify the energy density $\rho$ and pressure $p$ by
\begin{align*}
\tau_0^{eff}&=-\frac{1}{3!}\rho\epsilon_{0ijk}e^i\wedge e^j\wedge e^k,\\
\tau_i^{eff}&=-\frac{1}{2}p\epsilon_{0ijk}e^0\wedge e^j\wedge e^k.
\end{align*} 
Hence
\begin{align}\label{rho}
\rho&=-\frac{3}{2}\alpha_1\left(h^2-f^2+2Hh\right)\notag\\
&+3\omega\gamma\left[\frac{1}{2}\alpha_1+\alpha_2\left(\omega^2+\frac{k}{a^2}-f^2\right)\right],\\
\label{presion}
p&=\alpha_1\left[\dot{h}+Hh+\frac{1}{2}\left(h^2-f^2+2Hh\right)\right]\notag\\
&-2\omega\gamma\left[\frac{1}{2}\alpha_1+\alpha_2\left(\dot{\omega}+H\omega\right)\right]\notag\\
&+\left(\dot{\gamma}+\frac{1}{2}\gamma\dot{\Phi}\right)\left[\frac{1}{2}\alpha_1+\alpha_2\left(\omega^2+\frac{k}{a^2}-f^2\right)\right].
\end{align}
The identification satisfy $\dot{\rho}+3H\left(\rho+p\right)=0$. Using equations of motion, one can show that
\begin{align*}
\rho\sim H^2+\frac{k}{a^2}+\text{const.}
\end{align*}
For $u_\pm<0$, this function remains finite at the begining.
From expressions \eqref{rho} and \eqref{presion} one can see that the induced energy density and pressure are not positive definite quantities. In fact, due to the presence of torsion and extra dimensional fields, the universe solutions undergoes through accelerating expansive age with no need of extra fluids of exotic class. 

%_-----------------------------------------------------------------------------------------

\section{Discussion and Conclusions}\label{conclusions}
It is known that KK theory, in its simplest version, give rise matter fields a manifestation of extra dimensional metric degrees of freedom. The kind of matter that arises in the reduced spacetime is subordinated to the topology taken for the extra compact manifold. In general, those fields will gravitate and modulate the behaviour of spacetime in its time evolution. We have constructed the simplest model in five dimensional spacetime that allow the presence of torsion in the KK decomposition. In the cosmological model, the induced matter present due to an extra compact section taken to be $S^1$ are the usual scalar field --a metric degree of freedom-- and a vector field that comes from the dimensional reduction of a general spin connection. Torsion exists both in the reduced manifold and in the higher dimensional and indeed, it participate in the effective energy-momentum form that gravitates as a perfect fluid. Since the sign of the energy density nor of the presures in the fluid are in no way guaranteed, it is interpreted as a class of matter of exotic characteristics. Exact solution for a five dimensional vacuum are studied. Accelerating, and bouncing and inflationary universes are possible by the equations of motion.\\
There are periods of time where $f^2(t)$ becomes negative. As we can see from Eq.~\eqref{homotorsion}, $f$ corresponds to the completely antisymmetric part of torsion and is unseen by classical particles following geodesics along spacetime. It does not couple to spin-0 or spin-1 bosons, but it does appear as an effective interaction term in the Dirac Lagrangian writen in a curved spacetime \cite{Carroll:1994dq}
\begin{align*}
\mathcal{L}_{\text{Interaction}}&=-\frac{i}{8}T_{\alpha\mu\nu}\bar{\Psi}\Gamma^{\alpha\mu\nu}\Psi\\
&=\frac{3}{2}f\bar{\Psi}\Gamma^0\Gamma^5\Psi.
\end{align*}
It couples to the chiral current. For imaginary values of $f(t)$, the corresponding violation of current conservation can be interpreted as particle creation. Similar results where found in \cite{Toloza:2013wi} for another cosmological model with torsion that share some of the couplings of a reduced action. 
\begin{acknowledgments}
The authors would like to thank O. Miskovic, A. Toloza and J. Zanelli for enlightening discussions in this work.
\end{acknowledgments}
%_-----------------------------------------------------------------------------------------


\appendix
\section{Riemann-Cartan Geometry}\label{Riemann-Cartan}
In Riemann--Cartan geometry, both the vielbein $e^A$ and the spin connection $\omega^{AB}$ are independent fields. We will drop the hats since we will not deal with the KK decomposition. Stationary variations of a gravity actions must be taken with respect to both fields, allowing in general, the presence of torsion. The spin connection, however, can be decomposed in a Riemannian part $\tilde{\omega}^{AB}$ that satisfies $\tilde{\mbox{D}}e^A=0$, and a contorsion $\kappa^{AB}=-\kappa^{BA}$, such that 
\begin{equation}\label{spin separation}
  \omega^{AB}=\tilde{\omega}^{AB}+\kappa^{AB}.
\end{equation} 
Therefore, the torsion 2-form defined as the covariant derivative of the vielbein with respect to the total spin connection, is
\begin{equation}
  T^A=\kappa^A_{\ B}\wedge e^B.
\end{equation}
On the other hand, the curvature 2-form also suffers corrections with respect to the Riemannian one due to the presence of torsional degrees of freedom. This can be seen explictly by taking the definition of curvature $R^{AB}=\mbox{d}\omega^{AB}+\omega^a_{\ C}\wedge\omega^{CB}$ and \eqref{spin separation} to find
\begin{equation}\label{curvature decomp}
  R^{AB}=\tilde{R}^{AB}+\kappa^A_{\ C}\wedge\kappa^{CB}+\tilde{\mbox{D}}\kappa^{AB},
\end{equation}
where $\tilde{R}^{AB}$ is the Riemannian curvature built with $\tilde{\omega}^{AB}$.


%_-----------------------------------------------------------------------------------------

\section{Isotropic--Homogeneus Ansatz}\label{homotropic}
The cosmological principle demands the spacial section of spacetime to be isotropic and homogeneus. This means that the fields involve in the model must be compatible with this assumption. A spacetime is an isotropic spacetime with respect to certain point $P$ if after a rotation with respect to an axis that pass through $P$, all the geometrical properties remain invariant, thus the spacetime looks the same in all directions. By homogeneity we understand that spacetime looks the same from every point $P$. These two assumptions are translated in the Killing equations for the fields, which are the vanishing of the Lie derivatives of the fields along the direction of the vector generators of symmetries $\{\zeta^\lambda_{i}\}$. In particular the Killing equations for the metric tensor and the torsion tensor are
\begin{align}
  \label{killing metric}
  \text{\textsterling}_i g_{\mu\nu}&=\zeta^\lambda_i\partial_\lambda g_{\mu\nu}+\partial_\mu\zeta^\lambda_i g_{\lambda\nu}+\partial_\nu\zeta^\lambda_i g_{\mu\lambda}=0,\\
  \text{\textsterling}_i T^\alpha_{\ \mu\nu}&=\zeta^\lambda_i\partial_\lambda T^\alpha_{\ \mu\nu}-\zeta^\alpha_i\partial_\lambda T^\lambda_{\ \mu\nu}
  +\zeta^\lambda_i\partial_\mu T^\alpha_{\ \lambda\nu}\notag\\
  &\quad +\zeta^\lambda_i\partial_\nu T^\alpha_{\ \mu\lambda}=0,
\end{align}
where $T^\alpha_{\ \mu\nu}$ are the components of the torsion 2-form defined by $T^a=\frac{1}{2}e^a_{\ \alpha}T^\alpha_{\ \mu\nu}\mbox{d}x^\mu\wedge\mbox{d}x^\nu$. The same must hold for the new fields
\begin{align}
  \label{killing vector}
  \text{\textsterling}_i A_\mu&=\zeta^\lambda_i\partial_\lambda A_\mu+\zeta^\lambda_i\partial_\mu A_\lambda=0,\\
  \text{\textsterling}_i \phi &=\zeta^\lambda_i\partial_\lambda\phi=0,
\end{align}
and also for the components (tensors) $\alpha_{\mu\nu}=-\alpha_{\nu\mu}$, $\beta_{\mu\nu}$ and $\gamma_\mu$, defined such that
\begin{align*}
  \alpha^{ab}&=E^{a\mu}E^{b\nu}\alpha_{\mu\nu},\\
  \beta^a&=E^{a\mu}\beta_{\mu\nu}\mbox{d}x^\nu,\\
  \gamma^a&=E^{a\mu}\gamma_\mu,
\end{align*} 
and whose Killing equations are analogous to the tensorial one, \eqref{killing metric} and vectorial one,\eqref{killing vector}.\\
The set of Killing vectors $\{\zeta^\lambda_i\}$ are, on the one hand, the generators of spatial rotations in three dimensions $SO(3)$, $\xi_i=\epsilon_{ijk}x_j\partial_k$, and the Killing vectors assosiated with spacial translations $\mathcal{P}_i=\sqrt{1-kr^2}\partial_i$, that satisfy the algebra
\begin{align}
  \left[\xi_i,\xi_j\right]&=\epsilon_{ijk}\xi_k,\\
  \left[\mathcal{P}_i,\mathcal{P}_j\right]&=-k\epsilon_{ijk}\xi_k,\\
  \left[\xi_i,\mathcal{P}_j\right]&=\epsilon_{ijk}\xi_k.
\end{align}
These requirements on the fields are translated to a set of first order differential equations whose most general solutions determines our ansatz structure \eqref{vielbein cosmo} to \eqref{gamma cosmo}. The fields strengths for isotropic--homogeneus vielbein and spin connection are written next. The curvature components are
\begin{align}
  R^{0i}&=\left(\dot{\omega}+H\omega\right)e^0\wedge e^i+f\omega\epsilon^i_{\ jk}e^j\wedge e^k,\\
  R^{ij}&=\left(\omega^2+\frac{k}{a^2}-f^2\right)e^i\wedge e^j
  -\left(\dot{f}+Hf\right)\epsilon^{ij}_{\ \ k}e^0\wedge e^k,
\end{align}
and the torsion components
\begin{equation}\label{homotorsion}
  T^0=0,\quad T^i=-he^0\wedge e^i+f\epsilon^i_{\ jk}e^j\wedge e^k,
\end{equation}
where $h(t)$ and $H(t)$ were defined in Section \ref{5EGB}.

%_-----------------------------------------------------------------------------------------
\section{Time Dependence of the Solutions}\label{solutions t}
We supply here the time dependent expressions of the cosmological solutions for the significantly differents values of $u_\pm$ in terms of the dimensionless parameter $\tau=\sqrt{|u_\pm|}(t-t_0)$.\\
\newline
$\bullet\ u_\pm>0$:
\begin{align}
  \gamma(t)&=-\sqrt{u_\pm}\cot\tau,\\
  \phi(t)&=\phi_0\sin^2\tau\exp\left[\frac{\mp\sqrt{6\Delta}}{4\alpha_2}\cos^2\tau\right],\\
  a(t)&=\frac{a_0}{\sqrt{u_\pm}}|\cos\tau|\exp\left[\frac{\pm\sqrt{6\Delta}}{24\alpha_2}\cos^2\tau\right],\\
  f^2(t)&=u_\pm\frac{k}{a_0^2}\sec^2\tau\exp\left[\frac{\mp\sqrt{6\Delta}}{12\alpha_2u_\pm}\cos^2\tau\right]\notag\\
  &+u_\pm\tan^2\tau-Y_\pm.
\end{align}
\\
\medskip
$\bullet\ u_\pm=0$:
\begin{align}
  \gamma(t)&=0,\\
  \phi(t)&=\phi_0\exp\left[\frac{\mp\sqrt{6\Delta}}{4\alpha_2}(t-t_0)^2\right],\\
  a(t)&=a_0|t-t_0|\exp\left[\frac{\pm\sqrt{6\Delta}} {24\alpha_2}(t-t_0)^2\right],\\
  f^2(t)&=\frac{k}{a_0^2(t-t_0)^2}\exp\left[\frac{\mp\sqrt{6\Delta}}{12\alpha_2}(t-t_0)^2\right]\notag\\
  &+(t-t_0)^{-2}-Y_\pm.
\end{align}
\\
\medskip
$\bullet\ u_\pm<0$:
\begin{align}
  \gamma(t)&=-\sqrt{-u_\pm}\coth\tau,\\
  \phi(t)&=\phi_0\sinh^2\tau\exp\left[\frac{\mp\sqrt{6\Delta}}{4\alpha_2u_\pm}\cosh^2\tau\right],\\
  a(t)&=\frac{a_0}{\sqrt{-u_\pm}}\cosh\tau\exp\left[\frac{\pm\sqrt{6\Delta}} {24\alpha_2u_\pm}\cosh^2\tau\right],\\
  f^2(t)&=-u_\pm\frac{k}{a_0^2}\cosh^{-2}\tau\exp\left[\frac{\mp\sqrt{6\Delta}}{12\alpha_2u_\pm}\cosh^2\tau\right]\notag\\
  &-u_\pm\tanh^2\tau-Y_\pm.
\end{align}
%%%%%%%%% BIBLIOGRAPHY %%%%%%%%%
%%\bibliographystyle{apsrev4-1}
\bibliography{References.bib}

\end{document}

