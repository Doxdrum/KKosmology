\documentclass[aps,prd,12pt,superscriptaddress,showpacs,showkeys,longbibliography,reprint,nofootinbib]{revtex4-1}
\usepackage{calrsfs} 
\usepackage{amsmath,amsthm,latexsym,amssymb,amsfonts}
\usepackage{xcolor}
\usepackage{textcomp}
\usepackage{float}
\usepackage[%
  colorlinks=true,
  urlcolor=blue,
  linkcolor=blue,
  citecolor=blue
]{hyperref}
\usepackage{etoolbox}
\usepackage{breqn}

%% \makeatletter
%% \let\cat@comma@active\@empty
%% \makeatother

%------------------
%--------- Definitions
%------------------
\usepackage{mathtools,amsmath,amssymb,amsfonts,dsfont,mathrsfs,amsthm,bm,latexsym}
\usepackage{graphicx}
\usepackage{centernot}
\usepackage{xcolor}
\usepackage{comment}
%% \usepackage{hyperref}
%% \hypersetup{linktocpage,colorlinks=true,urlcolor=blue!80!red,linkcolor=blue,citecolor=red}
\usepackage{feynmf}
\usepackage{siunitx}
\usepackage{array}
\newcolumntype{L}[1]{>{\raggedright\let\newline\\\arraybackslash\hspace{0pt}}m{#1}}
\newcolumntype{C}[1]{>{\centering\let\newline\\\arraybackslash\hspace{0pt}}m{#1}}
\newcolumntype{R}[1]{>{\raggedleft\let\newline\\\arraybackslash\hspace{0pt}}m{#1}}
%% \usepackage{ulem}
\usepackage{tikz}
\usepackage{braket}
\usepackage{xparse}
\usetikzlibrary{shapes,
  snakes,
  decorations.pathmorphing,
  decorations.markings,
  calc,
  shadows.blur,
  shadings}
\usepackage[framemethod=tikz]{mdframed}

\makeatletter
% gluon decoration (based on the original coil decoration)
\pgfdeclaredecoration{gluon}{coil}
{
  \state{coil}[switch if less than=%
    0.5\pgfdecorationsegmentlength+%>
    \pgfdecorationsegmentaspect\pgfdecorationsegmentamplitude+%
    \pgfdecorationsegmentaspect\pgfdecorationsegmentamplitude to last,
               width=+\pgfdecorationsegmentlength]
  {
    \pgfpathcurveto
    {\pgfpoint@oncoil{0    }{ 0.555}{1}}
    {\pgfpoint@oncoil{0.445}{ 1    }{2}}
    {\pgfpoint@oncoil{1    }{ 1    }{3}}
    \pgfpathcurveto
    {\pgfpoint@oncoil{1.555}{ 1    }{4}}
    {\pgfpoint@oncoil{2    }{ 0.555}{5}}
    {\pgfpoint@oncoil{2    }{ 0    }{6}}
    \pgfpathcurveto
    {\pgfpoint@oncoil{2    }{-0.555}{7}}
    {\pgfpoint@oncoil{1.555}{-1    }{8}}
    {\pgfpoint@oncoil{1    }{-1    }{9}}
    \pgfpathcurveto
    {\pgfpoint@oncoil{0.445}{-1    }{10}}
    {\pgfpoint@oncoil{0    }{-0.555}{11}}
    {\pgfpoint@oncoil{0    }{ 0    }{12}}
  }
  \state{last}[next state=final]
  {
    \pgfpathcurveto
    {\pgfpoint@oncoil{0    }{ 0.555}{1}}
    {\pgfpoint@oncoil{0.445}{ 1    }{2}}
    {\pgfpoint@oncoil{1    }{ 1    }{3}}
    \pgfpathcurveto
    {\pgfpoint@oncoil{1.555}{ 1    }{4}}
    {\pgfpoint@oncoil{2    }{ 0.555}{5}}
    {\pgfpoint@oncoil{2    }{ 0    }{6}}
  }
  \state{final}{}
}

\def\pgfpoint@oncoil#1#2#3{%
  \pgf@x=#1\pgfdecorationsegmentamplitude%
  \pgf@x=\pgfdecorationsegmentaspect\pgf@x%
  \pgf@y=#2\pgfdecorationsegmentamplitude%
  \pgf@xa=0.083333333333\pgfdecorationsegmentlength%
  \advance\pgf@x by#3\pgf@xa%
}
\makeatother

\tikzset{
  boson/.style={decorate,decoration={gluon,segment length=9pt,aspect=0}},
  % style to apply some styles to each segment of a path
  on each segment/.style={
    decorate,
    decoration={
      show path construction,
      moveto code={},
      lineto code={
        \path [#1]
        (\tikzinputsegmentfirst) -- (\tikzinputsegmentlast);
      },
      curveto code={
        \path [#1] (\tikzinputsegmentfirst)
        .. controls
        (\tikzinputsegmentsupporta) and (\tikzinputsegmentsupportb)
        ..
        (\tikzinputsegmentlast);
      },
      closepath code={
        \path [#1]
        (\tikzinputsegmentfirst) -- (\tikzinputsegmentlast);
      },
    },
  },
  % style to add an arrow in the middle of a path
  mid arrow/.style={postaction={decorate,decoration={
        markings,
        mark=at position .5 with {\arrow[#1]{stealth}}
      }}},
}
\tikzstyle fermion=[draw,postaction={on each segment={mid arrow}}]


%-------------------------------Theorems
\newtheorem{Def}{Definition}[section]
\newtheorem{Thm}[Def]{Theorem}
\newtheorem{Lem}[Def]{Lemma}
\newtheorem{Pos}[Def]{Postulate}
\newtheorem{Exa}[Def]{Example}
\newtheorem{Cor}[Def]{Corrolary}
\newtheorem{Pro}[Def]{Proposition}

%---------------------------------New commands
\newcommand{\A}{\mathcal{A}} %% This is equivalent to \Ag (no args)
\newcommand{\abs}[1]{\left|{#1}\right|}
\DeclareMathOperator{\ad}{ad}
\DeclareMathOperator{\Ad}{Ad}
\DeclareDocumentCommand{\Ag}{ s o }{ \IfBooleanTF{#1}
    { \IfValueTF{#2}{ \bm{\mathcal{A}}_{(#2)} }{ \bm{\mathcal{A}} } }
    { \IfValueTF{#2}{    {\mathcal{A}}_{(#2)} }{    {\mathcal{A}} } } }
%% \newcommand{\Ag}{\mathcal{A}_{(1)}}
%% \newcommand{\Agf}{\boldsymbol{\mathcal{A}}}
\DeclareDocumentCommand{\Af}{ s o }{ \IfBooleanTF{#1}
    { \IfValueTF{#2}{ \boldsymbol{A}_{(#2)} }{ \boldsymbol{A} } }
    { \IfValueTF{#2}{            {A}_{(#2)} }{            {A} } } }
%% \newcommand{\Af}{ {\mathbf{A}} }
%% \newcommand{\AF}[1]{ {\mathbf{A}}_{({#1})} }
\newcommand{\hAf}{ \hat{\mathbf{A}}_{(1)} }
\newcommand{\hAF}[1]{ \hat{\mathbf{A}}_{(#1)} }
\newcommand{\bboxed}[1]{{\color{red}{\boxed{\boxed{\textcolor{black}{#1}}}}}}
\newcommand{\C}{\mathbb{C}}
\newcommand{\Cl}{\mathcal{C}\!\ell}
\newcommand{\cdf}[1][]{{\boldsymbol{\mathcal{D}}}{#1}}
\newcommand{\covd}{\mathcal{D}}
\newcommand{\D}{\mathscr{D}}
\newcommand{\df}[1][]{\mathbf{d}{#1}}
\newcommand{\dfd}{{\mathbf{d}}^\dag\!}
\newcommand\ded{\mathrm{d}^\dag}
\DeclareMathOperator{\End}{End}
\newcommand{\ele}[2][]{\frac{d}{dt}\left(\frac{\partial\mathcal{L}}{\partial \dot{#2}^{#1}}\right) - \frac{\partial\mathcal{L}}{\partial {#2}^{#1}}}
\newcommand{\fele}[2][]{\partial_\mu \left(\frac{\delta\mathcal{L}}{\delta\left(\partial_\mu{#2}^{#1}\right)}\right) - \frac{\delta\mathcal{L}}{\delta {#2}^{#1}}}
\newcommand{\vb}[1]{\vec{e}_{#1}}
%% \newcommand{\fb}[1]{\widetilde{e}{}^{ #1}}
\newcommand{\fb}[1]{\widetilde{e}{}^{ #1}}
\newcommand\fder[3][]{\frac{\delta^{#1}{#2}}{\delta {#3}^{#1}}}
\newcommand\fdern[4][]{\frac{\delta^{#1}{#2}}{\delta {#3} \cdots \delta {#4}}}
%% \newcommand{\Fc}{\mathcal{F}}
%% \newcommand{\F}{\boldsymbol{\mathcal{F}}}
%% \newcommand{\Fg}{\boldsymbol{\mathcal{F}}_{(2)}}
\DeclareDocumentCommand{\Fg}{ s o }{ \IfBooleanTF{#1}
    { \IfValueTF{#2}{ \bm{\mathcal{F}}_{(#2)} }{ \bm{\mathcal{F}} } }
    { \IfValueTF{#2}{    {\mathcal{F}}_{(#2)} }{    {\mathcal{F}} } } }
%% \newcommand{\Ff}{{\mathbf{F}}}
\DeclareDocumentCommand{\Ff}{ s o }{ \IfBooleanTF{#1}
    { \IfValueTF{#2}{ \boldsymbol{F}_{(#2)} }{ \boldsymbol{F} } }
    { \IfValueTF{#2}{            {F}_{(#2)} }{            {F} } } }
%% \DeclareDocumentCommand{\Ff}{ o }{ \IfValueTF{#1}{ F_{(#1)} }{ F } }
\newcommand{\FF}[1]{{\mathbf{F}}_{(#1)}}
\newcommand{\hFF}[1]{{\mathbf{\hat{F}}}_{(#1)}}
\newcommand{\fy}{\centernot}
\newcommand{\G}{\mathscr{G}}
\newcommand{\ga}{\gamma}
\newcommand{\gf}{\boldsymbol{\gamma}}
\newcommand{\Ga}{\Gamma}
\newcommand{\Ha}{\mathscr{H}}
\newcommand{\He}{\mathbb{H}}
\newcommand{\Hi}{\mathcal{H}}
\newcommand{\Hint}{\underline{\sc Hint:} }
\DeclareMathOperator{\hs}{\star\!\!}
\DeclareMathOperator{\st}{\star}
\newcommand{\J}{\mathscr{J}}
\newcommand{\K}{\mathbb{K}}
\newcommand{\KK}{Kaluza--Klein\ }
\newcommand{\Lag}{\mathscr{L}}
\newcommand{\Li}{\mathcal{L}}
\newcommand{\La}[1][]{\triangle_{#1}}
\newcommand{\Lap}{\nabla^2}
\newcommand{\lr}[1]{\stackrel{\leftrightarrow}{#1}}
\newcommand{\M}{\ensuremath{\mathscr{M}}}
\DeclareMathOperator{\Mat}{Mat}
\newcommand{\Mi}{\mathcal{M}}
\newcommand{\MN}{Maldacena-N\'u\~nez\ }
\newcommand{\N}{\ensuremath{\mathscr{N}}}
\newcommand{\Na}{\mathbb{N}}
\newcommand{\No}{\mathcal{N}}
\newcommand{\norm}[1]{\left\|#1\right\|}
\newcommand{\Op}{\mathcal{O}}
\newcommand{\Or}{\mathscr{O}}
\DeclareDocumentCommand{\PB}{ O{m} O{q} O{p} m m }{ \frac{ \partial #4 }{\partial {#2}^{#1} } \frac{ \partial #5 }{\partial {#3}_{#1} } - \frac{ \partial #4 }{\partial {#3}_{#1} } \frac{ \partial #5 }{\partial {#2}^{#1} } }
\newcommand\pder[3][]{\frac{\partial^{#1}{#2}}{\partial {#3}^{#1}}}
\newcommand\pdern[4][]{\frac{\partial^{#1}#2}{\partial #3\cdots\partial #4}}
\DeclareMathOperator{\Pfaff}{Pfaff}
\newcommand{\Qh}[1][]{\ensuremath{\hat{Q}_{#1}}}
\newcommand{\R}{\mathbb{R}}
\newcommand{\Ri}{\mathcal{R}}
%% \newcommand{\S}{\mathscr{S}}
\newcommand{\SM}{Standard Model {}}%\mathscr{S}}
\DeclareDocumentCommand\Te{o o m }{\mathcal{T}{}^{#1}_{#2}(#3)}
\newcommand{\T}{\mathscr{T}}
\newcommand\vdj[1]{\left< \Delta J \right>_{#1}}
\newcommand{\w}{{\scriptstyle\wedge}\!}
\newcommand{\we}{{\scriptstyle\wedge}}
\newcommand{\Z}{\mathbb{Z}}

\newcommand{\dbar}[1]{\ensuremath{\mathchar'26\mkern-12mu \mathrm{d}^{#1}}\!}


\DeclareDocumentCommand{\BM}{ s }{ \IfBooleanTF{#1} {\hat{\bm{M}}}{\bm{M}} }
\DeclareDocumentCommand{\BN}{ s }{ \IfBooleanTF{#1} {\hat{\bm{N}}}{\bm{N}} }
\DeclareDocumentCommand{\BP}{ s }{ \IfBooleanTF{#1} {\hat{\bm{P}}}{\bm{P}} }
\DeclareDocumentCommand{\BQ}{ s }{ \IfBooleanTF{#1} {\hat{\bm{Q}}}{\bm{Q}} }
\DeclareDocumentCommand{\BR}{ s }{ \IfBooleanTF{#1} {\hat{\bm{R}}}{\bm{R}} }
\DeclareDocumentCommand{\BS}{ s }{ \IfBooleanTF{#1} {\hat{\bm{S}}}{\bm{S}} }
\DeclareDocumentCommand{\BU}{ s }{ \IfBooleanTF{#1} {\hat{\bm{U}}}{\bm{U}} }
\DeclareDocumentCommand{\BV}{ s }{ \IfBooleanTF{#1} {\hat{\bm{V}}}{\bm{V}} }

%--------------------------- New Greek
\newcommand{\tht}{\ensuremath{\theta}}
\newcommand{\bet}{\ensuremath{\bar{\eta}}}
\newcommand{\bps}{\ensuremath{\bar{\psi}}}
\newcommand{\bc}{\ensuremath{\bar{\chi}}}
\newcommand{\Bps}{\ensuremath{\bar{\Psi}}}
\newcommand{\Bx}{\ensuremath{\bar{\Xi}}}
\newcommand{\bph}{\ensuremath{\bar{\phi}}}
\newcommand{\vph}{\ensuremath{\varphi}}
\newcommand{\bvph}{\ensuremath{\bar{\varphi}}}
\newcommand{\bth}{\ensuremath{\bar{\theta}}}
\newcommand{\hph}{\ensuremath{\hat{\phi}}}
\newcommand\bml{\bm{\lambda}}
\newcommand\bmk{\bm{\kappa}}

\newcommand{\bs}[1]{\boldsymbol{#1}}


\renewcommand{\div}{{\mathbf{div}}}
\newcommand{\grad}{{\mathbf{grad}}}
\newcommand{\curl}{{\mathbf{curl}}}

%%---------------------------------------------
%%--------- Vielbein definitions
%%---------------------------------------------
\NewDocumentCommand\MyAc{ m }{#1}
%%------------------------------ Vielbein form
\DeclareDocumentCommand{\vif}{ t. t, t- s s m }{
  \RenewDocumentCommand\MyAc{ m }{##1}
  \IfBooleanT{#1}{\RenewDocumentCommand\MyAc{ m }{ \mathring{##1} } }
  \IfBooleanT{#2}{\RenewDocumentCommand\MyAc{ m }{ \tilde{##1} } }
  \IfBooleanT{#3}{\RenewDocumentCommand\MyAc{ m }{ \bar{##1} } }
  \IfBooleanTF{#4}
  { \IfBooleanTF{#5} { \hat{\MyAc{\boldsymbol{e}}}^{\hat{#6}} }{ \hat{\MyAc{\boldsymbol{e}}}^{{#6}} } }
  { \MyAc{\boldsymbol{e}}^{{#6}} } }
%%------------------------------ Vielbein components
\DeclareDocumentCommand{\vi}{ t. t, t- s s m m}{
  \RenewDocumentCommand\MyAc{ m }{##1}
  \IfBooleanT{#1}{\RenewDocumentCommand\MyAc{ m }{ \mathring{##1} } }
  \IfBooleanT{#2}{\RenewDocumentCommand\MyAc{ m }{ \tilde{##1} } }
  \IfBooleanT{#3}{\RenewDocumentCommand\MyAc{ m }{ \bar{##1} } }
  \IfBooleanTF{#4}
  { \IfBooleanTF{#5} { \hat{\MyAc{e}}^{\hat{#6}}_{\hat{#7}} }{ \hat{\MyAc{e}}^{#6}_{{#7}} } }
  { \MyAc{e}^{{#6}}_{{#7}} } }
%%------------------------------ Compatibility
\newcommand\VI[2]{\hat{e}^{\hat{#1}}_{\hat{#2}}}
\newcommand\VIF[1]{\hat{\boldsymbol{e}}^{\hat{#1}}}
\newcommand\VIN[2]{\hat{E}^{\hat{#1}}_{\hat{#2}}}
\newcommand\VINF[1]{\hat{\boldsymbol{E}}_{\hat{#1}}}
\newcommand\hvi[2]{\hat{e}^{{#1}}_{{#2}}}
\newcommand\hvin[2]{\hat{E}^{{#1}}_{{#2}}}
\newcommand\hvif[1]{\hat{\boldsymbol{e}}^{{#1}}}
\newcommand\hvinf[1]{\hat{\boldsymbol{E}}_{{#1}}}
%% \newcommand\vi[2]{e^{{#1}}_{{#2}}}
\newcommand\vin[2]{E^{{#1}}_{{#2}}}
%% \newcommand\vif[1]{\boldsymbol{e}^{{#1}}}
\newcommand\vinf[1]{\boldsymbol{E}_{{#1}}}
%% \newcommand\Vi[2]{e^{\hat{#1}}_{\hat{#2}}}
%% \newcommand\Vin[2]{E^{\hat{#1}}_{\hat{#2}}}


%%---------------------------------------------
%%--------- Connections
%%---------------------------------------------
%%------------------------------ Christoffel symbols
\DeclareDocumentCommand{\ct}{ t. t, t- s s m m m }{
  \RenewDocumentCommand\MyAc{ m }{##1}
  \IfBooleanT{#1}{\RenewDocumentCommand\MyAc{ m }{ \mathring{##1} } }
  \IfBooleanT{#2}{\RenewDocumentCommand\MyAc{ m }{ \tilde{##1} } }
  \IfBooleanT{#3}{\RenewDocumentCommand\MyAc{ m }{ \bar{##1} } }
  \IfBooleanTF{#4}
  { \IfBooleanTF{#5} { \hat{\MyAc{\Gamma}}_{{#6}}{}^{\hat{#7}}{}_{\hat{#8}} }{ \hat{\MyAc{\Gamma}}_{{#6}}{}^{{#7}}{}_{{#8}} } }
  { \MyAc{\Gamma}_{{#6}}{}^{{#7}}{}_{{#8}} } }
%%------------------------------ Spin form
\DeclareDocumentCommand{\spif}{ t. t, t- s s m m }{
  \RenewDocumentCommand\MyAc{ m }{##1}
  \IfBooleanT{#1}{\RenewDocumentCommand\MyAc{ m }{ \mathring{##1} } }
  \IfBooleanT{#2}{\RenewDocumentCommand\MyAc{ m }{ \tilde{##1} } }
  \IfBooleanT{#3}{\RenewDocumentCommand\MyAc{ m }{ \bar{##1} } }
  \IfBooleanTF{#4}
  { \IfBooleanTF{#5} { \hat{\MyAc{\boldsymbol{\omega}}}^{\hat{#6}}{}_{\hat{#7}} }{ \hat{\MyAc{\boldsymbol{\omega}}}^{{#6}}{}_{{#7}} } }
  { \MyAc{\boldsymbol{\omega}}^{{#6}}{}_{{#7}} } }
%%------------------------------ Spin components
\DeclareDocumentCommand{\spi}{ t. t, t- s s m m m }{
  \RenewDocumentCommand\MyAc{ m }{##1}
  \IfBooleanT{#1}{\RenewDocumentCommand\MyAc{ m }{ \mathring{##1} } }
  \IfBooleanT{#2}{\RenewDocumentCommand\MyAc{ m }{ \tilde{##1} } }
  \IfBooleanT{#3}{\RenewDocumentCommand\MyAc{ m }{ \bar{##1} } }
  \IfBooleanTF{#4}
  { \IfBooleanTF{#5} { \hat{\MyAc{{\omega}}}_{\hat{#6}}{}^{\hat{#7}}{}_{\hat{#8}} }{ \hat{\MyAc{{\omega}}}_{{#6}}{}^{{#7}}{}_{{#8}} } }
  { \MyAc{{\omega}}_{{#6}}{}^{{#7}}{}_{{#8}} } }
%%------------------------------ Connections Compatibility
\newcommand{\conn}[3]{\left(\Gamma_{#1}\right)^{#2}{}_{#3}}
\newcommand{\connn}[3]{\Gamma_{#1}{}^{#2}{}_{#3}}
\newcommand{\Connn}[3]{\hat{\Gamma}_{#1}{}^{#2}{}_{#3}}
\newcommand{\CONNN}[3]{\hat{\Gamma}_{\hat{#1}}{}^{\hat{#2}}{}_{\hat{#3}}}
\newcommand{\hcon}[3]{\hat{\Gamma}_{#1}{}^{#2}{}_{#3}}
\newcommand\SPI[1]{\hat{\omega}_{\hat{#1}}}
\newcommand\SPIF[2]{\hat{\boldsymbol{\omega}}^{\hat{#1}}{}_{\hat{#2}}}
%% \newcommand\Spi[1]{\omega_{\hat{#1}}}
\newcommand\spc[1]{\omega_{{#1}}}
\newcommand\tspi[1]{\tilde{\omega}_{{#1}}}
%%\newcommand\spif[2]{{\boldsymbol{\omega}}^{{#1}}{}_{{#2}}}
\newcommand{\spife}[2]{\mathring{{\boldsymbol{\omega}}}^{#1}{}_{#2}}
\newcommand\tspif[2]{{\tilde{\boldsymbol{\omega}}}^{{#1}}{}_{{#2}}}
\newcommand\hspi[1]{\hat{\omega}_{{#1}}}
\newcommand\hspif[2]{\hat{\boldsymbol{\omega}}^{{#1}}{}_{{#2}}}
\newcommand\hspife[2]{\hat{\mathring{\boldsymbol{\omega}}}^{{#1}}{}_{{#2}}}

%%---------------------------------------------
%%--------- Curvature definitions
%%---------------------------------------------
%%------------------------------ Curvature form
\DeclareDocumentCommand{\rif}{ t. t, t- s s m m }{
  \RenewDocumentCommand\MyAc{ m }{##1}
  \IfBooleanT{#1}{\RenewDocumentCommand\MyAc{ m }{ \mathring{##1} } }
  \IfBooleanT{#2}{\RenewDocumentCommand\MyAc{ m }{ \tilde{##1} } }
  \IfBooleanT{#3}{\RenewDocumentCommand\MyAc{ m }{ \bar{##1} } }
  \IfBooleanTF{#4}
  { \IfBooleanTF{#5} { \hat{\MyAc{\bm{\mathcal{R}}}}{}^{\hat{#6}}{}_{\hat{#7}} }{ \hat{\MyAc{\bm{\mathcal{R}}}}{}^{{#6}}{}_{{#7}} } }
  { \MyAc{\bm{\mathcal{R}}}{}^{{#6}}{}_{{#7}} } }
%%------------------------------ Curvature components
\DeclareDocumentCommand{\ri}{ t. t, t- s s m m m }{
  \RenewDocumentCommand\MyAc{ m }{##1}
  \IfBooleanT{#1}{\RenewDocumentCommand\MyAc{ m }{ \mathring{##1} } }
  \IfBooleanT{#2}{\RenewDocumentCommand\MyAc{ m }{ \tilde{##1} } }
  \IfBooleanT{#3}{\RenewDocumentCommand\MyAc{ m }{ \bar{##1} } }
  \IfBooleanTF{#4}
  { \IfBooleanTF{#5} { \hat{\MyAc{\mathcal{R}}}_{{#6}}{}^{\hat{#7}}{}_{\hat{#8}} }{ \hat{\MyAc{\mathcal{R}}}_{{#6}}{}^{{#7}}{}_{{#8}} } }
  { \MyAc{\mathcal{R}}_{{#6}}{}^{{#7}}{}_{{#8}} } }
%%------------------------------ Curvature Compatibility
%%%%%%%%% Beware of the inconsistency between
%%%%%%%%% \Rif and \RIF
\newcommand{\RIF}[2]{\hat{\bm{\mathcal{R}}}^{\hat{#1}}{}_{\hat{#2}}}
\newcommand{\hRif}[2]{\hat{\bm{\mathcal{R}}}^{{#1}}{}_{{#2}}}
\newcommand{\Rif}[2]{\bm{\mathcal{R}}^{{#1}}{}_{{#2}}}
\newcommand{\tRif}[2]{\tilde{\bm{\mathcal{R}}}^{{#1}}{}_{{#2}}}

%%---------------------------------------------
%%--------- Contorsion definitions
%%---------------------------------------------
%%------------------------------ Contorsion form
\DeclareDocumentCommand{\kf}{ t. t, t- s s m m }{
  \RenewDocumentCommand\MyAc{ m }{##1}
  \IfBooleanT{#1}{\RenewDocumentCommand\MyAc{ m }{ \mathring{##1} } }
  \IfBooleanT{#2}{\RenewDocumentCommand\MyAc{ m }{ \tilde{##1} } }
  \IfBooleanT{#3}{\RenewDocumentCommand\MyAc{ m }{ \bar{##1} } }
  \IfBooleanTF{#4}
  { \IfBooleanTF{#5} { \hat{\MyAc{\bm{\mathcal{K}}}}^{\hat{#6}}{}_{\hat{#7}} }{ \hat{\MyAc{\bm{\mathcal{K}}}}^{{#6}}{}_{{#7}} } }
  { \MyAc{\bm{\mathcal{K}}}^{{#6}}{}_{{#7}} } }
%%------------------------------ Contorsion components
\DeclareDocumentCommand{\ko}{ t. t, t- s s m m m }{
  \RenewDocumentCommand\MyAc{ m }{##1}
  \IfBooleanT{#1}{\RenewDocumentCommand\MyAc{ m }{ \mathring{##1} } }
  \IfBooleanT{#2}{\RenewDocumentCommand\MyAc{ m }{ \tilde{##1} } }
  \IfBooleanT{#3}{\RenewDocumentCommand\MyAc{ m }{ \bar{##1} } }
  \IfBooleanTF{#4}
  { \IfBooleanTF{#5} { \hat{\MyAc{\mathcal{K}}}_{{#6}}{}^{\hat{#7}}{}_{\hat{#8}} }{ \hat{\MyAc{\mathcal{K}}}_{{#6}}{}^{{#7}}{}_{{#8}} } }
  { \MyAc{\mathcal{K}}_{{#6}}{}^{{#7}}{}_{{#8}} } }
%%------------------------------ Contorsion Compatibility
\newcommand{\Kf}[2]{{\bm{\mathcal{K}}}^{#1}{}_{#2}}
\newcommand{\cont}[3]{\mathcal{K}_{#1}{}^{#2}{}_{#3}}
\newcommand{\contf}[2]{\bm{\mathcal{K}}^{#1}{}_{#2}}
\newcommand{\hcont}[3]{\hat{\mathcal{K}}_{#1}{}^{#2}{}_{#3}}
\newcommand{\hcontf}[2]{\hat{\bm{\mathcal{K}}}^{#1}{}_{#2}}
\newcommand{\CONT}[3]{\hat{\mathcal{K}}_{\hat{#1}}{}^{\hat{#2}}{}_{\hat{#3}}}
\newcommand{\CONTU}[3]{\hat{\mathcal{K}}_{\hat{#1}}{}^{\hat{#2}\hat{#3}}}
\newcommand{\Contu}[3]{\hat{\mathcal{K}}_{{#1}}{}^{{#2}{#3}}}
\newcommand{\contu}[3]{{\mathcal{K}}_{{#1}}{}^{{#2}{#3}}}
\newcommand{\CONTF}[2]{\hat{\bm{\mathcal{K}}}^{\hat{#1}}{}_{\hat{#2}}}

%%---------------------------------------------
%%--------- Torsion definitions
%%---------------------------------------------
%%------------------------------ Torsion form
\DeclareDocumentCommand{\tf}{ t. t, t- s s m }{
  \RenewDocumentCommand\MyAc{ m }{##1}
  \IfBooleanT{#1}{\RenewDocumentCommand\MyAc{ m }{ \mathring{##1} } }
  \IfBooleanT{#2}{\RenewDocumentCommand\MyAc{ m }{ \tilde{##1} } }
  \IfBooleanT{#3}{\RenewDocumentCommand\MyAc{ m }{ \bar{##1} } }
  \IfBooleanTF{#4}
  { \IfBooleanTF{#5} { \hat{\MyAc{\bm{\mathcal{T}}}}^{\hat{#6}} }{ \hat{\MyAc{\bm{\mathcal{T}}}}^{{#6}} } }
  { \MyAc{\bm{\mathcal{T}}}^{{#6}} } }
%%------------------------------ Torsion Compatibility
\newcommand{\tor}{\mathcal{T}}
\newcommand{\tors}[3]{\mathcal{T}{}_{#1}{}^{#2}{}_{#3}}
\newcommand{\Tors}[3]{\hat{\mathcal{T}}{}_{#1}{}^{#2}{}_{#3}}
\newcommand{\TORS}[3]{\hat{\mathcal{T}}{}_{\hat{#1}}{}^{\hat{#2}}{}_{\hat{#3}}}
\newcommand{\torss}[3]{T_{#1}{}^{#2}{}_{#3}}
\newcommand{\Tf}[1]{\bm{\mathcal{T}}^{#1}}
\newcommand{\hTf}[1]{\hat{\bm{\mathcal{T}}}^{#1}}
\newcommand{\TF}[1]{\hat{\bm{\mathcal{T}}}^{\hat{#1}}}
\newcommand{\Tor}[2]{\mathcal{T}^{#1}{}_{#2}}

\newcommand\GAM[1]{{\gamma}^{\hat{#1}}}
%% \newcommand\Gam[1]{\gamma^{\hat{#1}}} 
\newcommand\gam[1]{\gamma^{{#1}}}
\newcommand\hgam[1]{\hat{\gamma}^{{#1}}}
%% %%%%%%%%%%%%%%%%%%
%% Problematic commands
%% with arXiv protocols
%% %%%%%%%%%%%%%%%%%%
\newcommand\NAB[1]{\hat{\nabla}_{\hat{#1}}}
\newcommand\Nab[1]{\nabla_{\hat{#1}}}
\newcommand\nab[1]{\nabla_{{#1}}}
\newcommand\PA[1]{\partial_{\hat{#1}}}
\newcommand\pa[1]{\partial_{{#1}}}
\newcommand\PAU[1]{\partial^{\hat{#1}}}
\newcommand\pau[1]{\partial^{{#1}}}
%% until here--- Change them before
%% trying to upload to arXiv
\newcommand\lf[1]{{\omega}^{{#1}}}
\newcommand\lft[1]{\hat{\omega}^{{#1}}}

\newcommand{\free}[1]{\mathring{#1}}
%% \newcommand{\ket}[1]{\left.\left|#1\right.\right>}
%% \newcommand{\bra}[1]{\left.\left<#1\right.\right|}
\renewcommand\bra[1]{\Bra{#1}}
\renewcommand\ket[1]{\Ket{#1}}
\newcommand{\bkt}[3]{\Braket{ {#1} | {#2} | {#3} } }
\newcommand{\bk}[2]{\Braket{ {#1} | {#2} } }
\newcommand{\comm}[2]{\left[#1,#2\right]}
\newcommand{\acomm}[2]{\left\{#1,#2\right\}}
\newcommand{\vev}[1]{\ensuremath{\left<#1\right>}}
\renewcommand{\set}[1]{\ensuremath{\Set{ #1 }}}

\newcommand{\relphantom}[1]{\mathrel{\phantom{#1}}}

\newcommand{\Riem}{\operatorname{Riem}}
\newcommand{\Ric}{\operatorname{Ric}}
\newcommand*{\diag}{\operatorname{diag}}
\newcommand{\id}{\operatorname{id}}
\newcommand{\tr}{\operatorname{tr}}
\newcommand{\Tr}{\operatorname{Tr}}
\newcommand{\Ker}{\operatorname{Ker}}
\renewcommand{\Im}{\operatorname{Im}}
\newcommand{\sgn}{\operatorname{sgn}}
\newcommand{\Ln}{\operatorname{Ln}}
\newcommand{\Ei}{\operatorname{Ei}}
\newcommand{\csch}{\operatorname{csch}}
\newcommand{\arcsinh}{\operatorname{arcsinh}}
\DeclareMathOperator\Br{Br}

\newcommand{\beq}{\begin{equation}}
\newcommand{\eeq}{\end{equation}}
\newcommand{\ber}{\begin{eqnarray}}
\newcommand{\eer}{\end{eqnarray}}

%% \renewcommand{\(}{\left(}
%% \renewcommand{\)}{\right)}
%% \renewcommand{\[}{\left[}
%% \renewcommand{\]}{\right]}

\newcommand{\uf}[2][]{\ensuremath{u_{#1}\(\vec{#2}\)}}
\newcommand{\ufb}[2][]{\ensuremath{\bar{u}_{#1}\(\vec{#2}\)}}
\newcommand{\vf}[2][]{\ensuremath{v_{#1}\(\vec{#2}\)}}
\newcommand{\vfb}[2][]{\ensuremath{\bar{v}_{#1}\(\vec{#2}\)}}
\newcommand\pol[2][]{\ensuremath{\varepsilon_{#1}(\vec{#2})}}
\newcommand\polc[2][]{\ensuremath{\varepsilon^*_{#1}(\vec{#2})}}
\newcommand{\ann}[3]{\ensuremath{#1\(\vec{#2},#3\)}}
\newcommand{\cre}[3]{\ensuremath{#1^\dag\(\vec{#2},#3\)}}
%% \newcommand{\uf}[2]{\ensuremath{u\(\vec{#1},#2\)}}
%% \newcommand{\ufb}[2]{\ensuremath{\bar{u}\(\vec{#1},#2\)}}
%% \newcommand{\vf}[2]{\ensuremath{v\(\vec{#1},#2\)}}
%% \newcommand{\vfb}[2]{\ensuremath{\bar{v}\(\vec{#1},#2\)}}
%% \newcommand{\ann}[3]{\ensuremath{#1\(\vec{#2},#3\)}}
%% \newcommand{\cre}[3]{\ensuremath{#1^\dag\(\vec{#2},#3\)}}

\newcommand{\dif}{{\mathrm{d}}}
\newcommand{\difn}[1]{{\mathrm{d}}^{#1}}
\newcommand{\dn}[2]{{\mathrm{d}}^{#1}\!{#2}\;}
\newcommand{\dbarn}[2]{\dbar{#1}{#2}\;}
\newcommand*{\de}[1]{\mathop{\mathrm{d}#1}\nolimits}% differential, facultative argoment between square parentheses
\newcommand*{\desec}[1][]{\mathop{\mathrm{d^2}#1}\nolimits}% second differential, facultative argoment between square parentheses
\newcommand{\der}[2]{\frac{\de{#1}}{\de{#2}}}% first derivative 
%\newcommand{\pder}[2]{\frac{\pa{}#1}{\pa{}{#2}}}% first derivative 
\newcommand{\inlineder}[2]{\mathrm{d}{#1}/\mathrm{d}{#2}}% in-line first derivative
\newcommand{\dersec}[2]{\frac{{\desec[#1]}}{\de[{#2}^2]}}% second derivative
%% \newcommand{\dx}{\de[x]}% frequently used differentials
%% \newcommand{\dy}{\de[y]}
%\newcommand{\df}{\de[f]}

%------------------
%--------- Format
%------------------
\newcommand{\out}[1]{{\color{red} {#1}}}
\newcommand{\pro}[1]{{\color{blue!70!black} {#1}}}
\newcommand{\hl}[1]{{\color{red} \bfseries{#1}}}

\newcommand\UTFSM{Departamento de F\'isica, Universidad T\'{e}cnica Federico Santa Mar\'\i a, \\ Casilla 110-V, Valpara\'iso, Chile}
\newcommand\UTFSMmat{Departamento de Matem\'aticas, Universidad T\'{e}cnica Federico Santa Mar\'\i a, \\ Casilla 110-V, Valpara\'iso, Chile}
\newcommand\CCTVal{Centro Cient\'ifico Tecnol\'ogico de Valpara\'iso, \\ Casilla 110-V, Valpara\'\i so, Chile}
\newcommand\CFF{Centro de F\'isica Fundamental,  Universidad de los Andes,\\ 5101 M\'erida, Venezuela}
\newcommand\PUCV{Instituto de F\'isica, Pontificia Universidad Cat\'olica de Valpara\'iso,\\ Casilla 4059, Valpara\'iso, Chile}
\newcommand\CECS{Centro de Estudios Cient\'ificos (CECs), Casilla 1469, Valdivia, Chile}




\hypersetup{%
  pdftitle={Kaluza--Klein Cosmology from five-dimensional Lovelock--Cartan Theory},
  pdfauthor={Oscar Castillo-Felisola,}{Cristobal Corral,}{Simon del Pino.},
  pdfkeywords={Torsion,} {Cosmology,}{Kaluza--Klein reduction,}{Generalised Gravity.},
  pdflang={English}
}


%------------------
%--------- Document
%------------------
\begin{document}

\title{Kaluza--Klein Cosmology from five-dimensional Lovelock--Cartan Theory}

\author{Oscar \surname{Castillo-Felisola}}
\email{o.castillo.felisola@gmail.com}
\affiliation{\UTFSM.}
\affiliation{\CCTVal.}

\author{Cristobal \surname{Corral}}
\email{cristobal.corral@usm.cl}
\affiliation{\UTFSM.}
\affiliation{\CCTVal.}

\author{Sim\'on \surname{del~Pino}}
\email{simon.delpino.m@mail.pucv.cl}
\affiliation{\PUCV.}

%% \author{Francisca \surname{Ram\'irez}.}
%% \affiliation{\UTFSM.}

%% --------- Abstract
\begin{abstract}
  We study the Kaluza--Klein dimensional reduction of the Lovelock--Cartan theory on a five-dimensional manifold, where the compact dimension has $S^1$ topology. We find cosmological solutions of the Friedmann--Robertson--Walker class for the effective theory.
  %%In the five-dimensional vacuum, this theory allows a nontrivial torsion in contrast to the Einstein--Cartan theory.
  The torsional degrees of freedom induce a nonvanishing energy-momentum tensor in four dimensions. We find solutions describing accelerated expansion phases as well as contracting universes. The model shows a dynamical compactification of the extra dimension in some regions of the parameter space. 
\end{abstract}

\pacs{02.40.Ma,04.50.Kd,04.90.+e}
\keywords{Affine Gravity, Torsion, Generalised Gravity.}


\maketitle

\section{Introduction}

%The theory of General Relativity (GR) has been proven to be one of the most successful description of gravity nowadays. In this framework, the gravitational interaction is understood as the geometrical manifestation of the spacetime. 
The experimental status of General Relativity (GR), regarding the solar system tests~\cite{Will:2014kxa} and the detection of signals consistent with the merge of two black holes by the LIGO collaboration~\cite{Abbott:2016blz,Abbott:2016nmj}, have settled it as the most successful theory of gravity so far. However, the difficulties on finding out explanations for the so called \emph{dark sector} of the Universe,
% and a \emph{consistent} quantum theory of gravitation,
have driven the community to think that GR is not the ultimate gravitational theory. 
The \emph{dark sector} of the Universe is composed by two kinds of degrees of freedom, whose effects are detected at different scales. On the one hand, at galactic scales, the gravitational lensing produced by the local distribution of energy, suggests the presence of an exotic form of matter, unseen by the current light-based telescopes~\cite{Sofue:2000jx}, for this it is named dark matter. Such an abundance in the galaxy is also compatible with the velocity profile of stars at its outter regions and it is needed to achieve the current structures formation. This matter would interact mostly (if not only) through gravity. On the other hand, at cosmic scales, the experimental data obtained from Type Ia supernovae observations, indicates that our Universe is passing through a phase of accelerated expansion~\cite{Riess:1998cb}. This introduces the dark energy hypothesis to generate such a behavior. 

The shortcomings of GR on describing these phenomena, are the main motivation to look for new gravitational degrees of freedom. Among the possible extensions, higher dimensional models could shed some light on the nature of these new degrees of freedom. For instance, as it was shown in the early works of T.~Kaluza and O.~Klein, the existence of an extra dimension within the GR framework would give rise to a unified picture of gravity and electromagnetism, along with a spectrum of new heavy particles~\cite{Kaluza:1921tu,*Klein:1926tv}. This idea opened the possibility of a novel geometrical understanding of interactions, where the gauge group arise as a consequence of the topology of the spatial compact manifold in a higher dimensional spacetime. The idea of higher dimensions comes naturally in diverse physical models, as for example: supersymmetry and supergravity, string theory~\cite{Green:1987sp,*Green:1987mn}, and novel proposals by Arkani-Hammed, Dimopoulos and Dvali~\cite{ArkaniHamed:1998rs,*Antoniadis:1998ig}, and Randall and Sundrum~\cite{Randall:1999ee,*Randall:1999vf} models, as attempts of solving the hierarchy problem.

In four dimensions, the Einstein--Hilbert action with cosmological constant is the most general theory which leads to second order field equations for the metric. In higher dimensions, particular combinations of higher order terms in the curvature can be added to the gravitational action, giving rise to second order field equations for the metric. The most general theory in arbitrary dimensions, which preserves the features of the four-dimensional Einstein--Hilbert, is described by the Lanczos--Lovelock action~\cite{Lanczos:1938sf,Lovelock:1971yv}. Such a theory, has no ghosts~\cite{Zumino:1985dp} and it has the same degrees of freedom as the Einstein--Hilbert Lagrangian in arbitrary dimensions~\cite{Henneaux:1990au}.

The simplest possible extra term in the Lanczos--Lovelock action  is a quadratic construction of curvatures, called Gauss--Bonnet term, which reads
\begin{equation}\label{GB}
  \mathcal{L}_{\mathrm{GB}} = \dn{N}{x}\sqrt{-g}\left(\tilde{R}^2-4\tilde{R}_{\mu\nu}\tilde{R}^{\mu\nu}+\tilde{R}_{\alpha\beta\mu\nu}
  \tilde{R}^{\alpha\beta\mu\nu}\right),
\end{equation}
where $\tilde{R}_{\alpha\beta\mu\nu}$ is the Riemannian curvature of a manifold with metric $g_{\mu\nu}$ and $g$ its determinant. $\tilde{R}_{\mu\nu}$ and $\tilde{R}$ are the Ricci tensor and Ricci scalar, respectively.

The Gauss--Bonnet term, in four dimensions, adds no dynamics to the metric, since it represents a topological invariant proportional to the Euler characteristic class, which can be written locally as a boundary term. In dimensions higher than four it contributes to the field equations.
%% Equivalently, the previous $N$-dimensional Lagrangian can be written in exterior forms as
%% \begin{equation}\label{GB2}
%% \mathcal{L}_{GB} = \epsilon_{a_1...a_N} \tilde{R}^{a_1a_2}\wedge\tilde{R}^{a_3a_4}\wedge e^{a_5}\wedge...\wedge e^{a_N}.
%% \end{equation}
This Lagrangian was also identified as the low-energy correction for a spin two field in string theory~\cite{Zwiebach:1985uq}.
%%Thus, the consideration of further Lanczos--Lovelock terms in a higher dimensional manifold can be interpreted as the semiclassical corrections to the gravity action.

On the one hand, classically, wormholes must be supported by a kind energy compatible with the cosmological hypothesis. Their existence in extended models, implies the presence of degrees of freedom that could provide an explanation for the exotic matter/energy abundance in the Universe. In the Riemannian five-dimensional Lanczos--Lovelock theory, exact wormholes solutions have been found in vacuum~\cite{Dotti:2006cp,Dotti:2007az}. The same kind of solutions were found with matter~\cite{Mehdizadeh:2015jra}, in higher order Lanczos--Lovelock models~\cite{Mehdizadeh:2015dta}, and in higher order compactified Lanczos--Lovelock theories with torsion~\cite{Canfora:2008ka}. Additionally, the compactification of higher Lanczos--Lovelock terms has been considered in Refs.~\cite{MuellerHoissen:1985mm,MuellerHoissen:1989yv} and their cosmology in Refs.~\cite{MuellerHoissen:1985ij,Deruelle:1986iv,Deruelle:1989fj}.  

On the other hand, the renaissance of the Riemann--Cartan geometries came through the pioneer works of Sciama~\cite{Sciama:1962} and Kibble~\cite{Kibble:1961ba} on Poincar\'e gauge theories (see also Refs.~\cite{Hehl:1976kj,Blagojevic:2002du}). In such a framework, the torsion appears as the field strength of translations, sourced by the spin-current.
%In such a framework, the vielbein is interpreted as the gauge connection of translations and the torsion as its field strength. The formulation of gravity as a gauge theory has major benefits in the problem of a unified picture of interactions, since it allows to interpret it as arising through a localization procedure, where the spacetime is a Riemann--Cartan manifold. 
%Due to this, several authors have relaxed the torsion-free condition since, as E. Cartan argued, there is no a fundamental reason to assume a torsion-free condition {\it a priori}. 
Moreover, the vacuum predictions of its simplest formulation---the Einstein--Cartan theory---holds the experimental tests of GR. The cosmological consequences of non-Riemannian geometries has been widely studied in the literature (for a review, see Ref.~\cite{Puetzfeld:2004yg}). %For instance, the pioneer works of Kopczy\'nski showed how the introduction of a spinning dust, acting as a source of the curvature and torsion, avoids the appearance of cosmological singularities~\cite{Kopczynski:1972,Kopczynski:1973}. In the context of the early evolution of the Universe, the spin density can drive the inflationary phase within the Einstein--Cartan theory, along with the avoidance of singularities~\cite{Gasperini:1986mv}. It has been shown in Ref.~\cite{Shie:2008ms} that torsional degrees of freedom might produce the current phase of accelerated expansion of the Universe. Additionally, the nonminimal coupling of the Euler characteristic form with a scalar field in Riemann--Cartan spacetimes, allows the torsional degrees of freedom to behave as an exotic induced-matter content, customary for the phase of accelerated expansion~\cite{Toloza:2013wi}. 

In the framework of the Kaluza--Klein theories in higher dimensional Riemann--Cartan geometries, phenomenology of the extra dimensional torsion was found in Ref.~\cite{Kalinowski:1980da}, metric dependent torsional models of extra dimensions was studied in~\cite{Shankar:2012vd}, along with its consequences in cosmology~\cite{Chen:2009ep}. Compatification of higher dimensional Brans--Dicke models with torsion was considered in Ref.~\cite{German:1993bq}. Torsion-free black-hole solutions were found in~\cite{Aros:2007nn}, for first order compactified gravity.

The extension of the Lanczos--Lovelock theories with nonvanishing torsion is known as Lovelock--Cartan~\cite{Mardones:1990qc}. In that framework, we present a new class of cosmological solutions in five dimensions, where the compact dimension is $S^1$. The theory admits a nonvanishing torsion in vacuum due to the presence of the Gauss--Bonnet term, in contrast to the five-dimensional Einstein--Cartan theory. After the dimensional reduction, the extra degrees of freedom generate an energy density and pressure as four-dimensional matter fields. In some cases, the solutions avoid the appearance of initial singularities and drive the accelerated expansion of the Universe, while the radius of the compact manifold goes to zero.  

This work is organized as follows: In Sec.~\ref{KK} we introduce the torsional degrees of freedom compatible with the Kaluza--Klein geometry, and fix our notation and conventions. In Sec.~\ref{5EGB}, we study the dynamics of the general Lovelock--Cartan action in five-dimensional spacetime and its dimensional reduction. In Sec.~\ref{cosmos}, we look for cosmological solutions of the Friedmann--Robertson--Walker class with nontrivial torsion. Conclusions and remarks are given in Sec.~\ref{conclusions}. We also incorporate a number of appendices to make this work a self-contained article.


\section{Kaluza--Klein Ansatz in Riemann--Cartan Spacetimes\label{KK}}

In the following we will consider $M_N$ to be a $N$-dimensional differential manifold. Every quantity defined on $M_N$ will be denoted by hats $\hat{x}$. Capital Greek characters (coordinate indices) and capital Latin characters (Lorentz indices) run over the $N$ dimensions, while lower case run in the $(N-1)$-dimensional reduced manifold.

The differential structure on $M_N$ is determined by two fields. The spin connection 1-form, $\hat{\omega}^{AB} = \hat{\omega}^{AB}{}_{\Gamma}\,\text{d}\hat{x}^\Gamma$, describes the affine structure of the $N$-dimensional manifold, and the vielbein 1-form, $\hat{e}^A=\hat{e}^{A}{}_{\Gamma}\,\text{d}\hat{x}^\Gamma$, defines the metric structure of the same manifold through the relation $\hat{g}_{\Gamma\Delta} = \hat{e}^{A}{}_{\Gamma}\hat{e}^{B}{}_{\Delta}\hat{\eta}_{AB}$. The vielbein is the mapping that relates coordinate indices with the Lorentz indices.
In our convention, both Lorentzian metrics \(\hat{\eta}\) and \(\eta\) have positive signature, and the Levi-Civita symbols are such that
\[\hat{\epsilon}_{a_1 \cdots a_{N-1} N} \equiv \epsilon_{a_1 \cdots a_{N-1}},\]
%% \[\hat{\epsilon}_{0 1 \cdots N-1} = {\epsilon}_{0 1 \cdots N-1} = 1,\]
which allows us to define the reduced symbol by fixing the last index. Additionally, all Riemannian fields (torsion-free) will be explicited by a tilde, as it was done in Eq.~\eqref{GB} for the curvature.
%% In our convention, the flat Lorentzian metric will be $\hat{\eta}_{AB}=\diag(-,+,\cdots,+)$, and the $N$-dimensional Levi-Civita symbol $\hat{\epsilon}_{A_1 \cdots A_N}$ is reduced by defining
%% \begin{align*}
%% \hat{\epsilon}_{a_1 \cdots a_{N-1} N} \equiv \epsilon_{a_1 \cdots a_{N-1}},
%% \end{align*} 
%% since the coordinates of $M_N$ run from $0$ to $N-1$. Our convention fixes $\hat{\epsilon}_{01...N}=1$. Hereon, Riemannian fields (torsion-free) will be explicited by a tilde, as it was done in Eqs.~\eqref{GB} for the curvature.

Due to the topology of the manifold, we can expand the depenence of the fields on the extra coordinate, $z$, in Fourier series as  
\begin{align}\label{Fourier}
  \hat{\varrho}(\hat{x}^\Gamma)&=\sum_n\varrho_{(n)}(x)e^{i n z},
\end{align}
where $x$ denotes the coordinates of the $(N-1)$-manifold, collectively. Henceforth, we will focus in the $n=0$ mode of the expasion, also refered to as the low-energy sector.

\subsection{Metric and affine structure}

The Kaluza--Klein (KK) ansatz for the metric relies on the premise that the compact dimension of $M_N$ is orthogonal to the rest of the manifold at each point (see for example Ref.~\cite{Blagojevic:2002du}). This leads to the following metric structure
\begin{equation}
  \hat{g}_{\Gamma\Delta} =
  \begin{pmatrix}
    g_{\gamma\delta} +\frac{\hat{g}_{\gamma z}\hat{g}_{\delta z}}{\hat{g}_{zz}}&\hat{g}_{\gamma z}\\
    \hat{g}_{z\delta} & \hat{g}_{zz}
  \end{pmatrix}
  =
  \begin{pmatrix}
    g_{\gamma\delta} + \phi A_\gamma A_\delta&\phi A_\gamma\\
    \phi A_{\delta} & \phi
  \end{pmatrix},
\end{equation}
and its inverse
\begin{equation}
  \hat{g}^{\Gamma\Delta}=
  \begin{pmatrix}
    g^{\gamma\delta}&-A^\gamma\\
    -A^{\delta} & \phi^{-1}+A^2
  \end{pmatrix}.
\end{equation}
It introduces a scalar field $\phi$ and a vector field $A_\mu$ as new gravitational degrees of freedom. The vielbein which holds this structure for the metric has the following form
\begin{equation}
  \label{Dvielbein}
  \hat{e}^A{}_{\Gamma} =
  \begin{pmatrix}
    \hat{e}^a{}_{\gamma}& 0\\
    \hat{e}^N{}_{\gamma} & \hat{e}^N{}_{z}
  \end{pmatrix}
  =
  \begin{pmatrix}
    e^a{}_{\gamma}& 0\\
    \sqrt{\phi}A_\gamma & \sqrt{\phi}
  \end{pmatrix},
\end{equation}
and its inverse, $\hat{E}_A{}^{\Gamma}$, defined such that $\hat{E}_A{}^{\Gamma}\hat{e}^A{}_{\Delta}=\delta^\Gamma_{\Delta}$, reads
\begin{equation}
  \label{Dinversevielbein}
  \hat{E}_A{}^{\Gamma} =
  \begin{pmatrix}
    \hat{E}_a{}^{\gamma}& 0\\
    \hat{E}_a{}^{z} & \hat{E}_N{}^{z}
  \end{pmatrix}
  =
  \begin{pmatrix}
    E_a{}^{\gamma}& 0\\
    -A_a & \sqrt{\phi}^{-1}
  \end{pmatrix},
\end{equation}
and it shows that the compact manifold $S^1$ has its own tangent space, $T_pS^1$, at each point $p\in M_N$, and is independent of $T_pM_{N-1}$ in the sense that they do not mix. %It seems reasonable to think that if we want two manifolds to be orthogonal to one another, they will not mix vectors after the tangent mapping.

The vielbeins are defined modulo Lorentz transformations. Any transformed basis $\hat{e}^{\prime A} = \hat{\Lambda}^A{}_{B}\,\hat{e}^B$,
where $\hat{\Lambda}$ is a Lorentz matrix, is as suitable as $\hat{e}^A$, and therefore share the structure of Eq.~\eqref{Dvielbein}. The Lorentz transformations are then constrained by $\hat{\Lambda}^a{}_{N}=0$.

A Riemannian connection compatible with the \mbox{$N$-dimensional} vielbein~\eqref{Dvielbein}, is built under the premise that $\mbox{d}\hat{e}^A + \hat{\tilde{\omega}}^{A}{}_B \wedge \hat{e}^B = 0$. Thus, we find
\begin{equation}
  \label{DRiemannconnection}
  \begin{split}
    \hat{\tilde{\omega}}^{ab}&=\tilde{\omega}^{ab}-\frac{1}{2}\sqrt{\phi}F^{ab}\hat{e}^N,\\
    \hat{\tilde{\omega}}^{Na}&=\frac{1}{2}\sqrt{\phi}F^a_{\ \ l}e^l+\frac{1}{2}\partial^a\ln\phi\hat{e}^N,
  \end{split}
\end{equation}
where $\tilde{\omega}^{ab}$ is the Riemannian spin connection of the reduced manifold and $F_{ab}$ are the components of the field strength of $A_b$ defined as
\begin{equation}
  F=\text{d}A=\frac{1}{2}F_{ab}\, e^a\wedge e^b.
\end{equation}
The construction of the five-dimensional Einstein--Hilbert action by means of the spin connection~\eqref{DRiemannconnection}, leads to the original KK theory.

Nevertheless, in a Riemann--Cartan geometry $M_N$ is not entirely described by $\hat{\tilde{\omega}}^{AB}$, but by a more general spin connection, independent of the metric degrees of freedom (see Appendix~\ref{Riemann-Cartan} for details). We can assume a general connection 1-form of the same type (regarding its $M_{N-1}\times S^1$ decomposition) as in Eq.~\eqref{DRiemannconnection}. Thus, the most general spin connection on $M_N$ compatible with the KK decomposition is given by 
\begin{equation}\label{Nconnection}
  \hat{\omega}^{AB} \equiv
  \begin{pmatrix}
    \omega^{ab}+\alpha^{ab}\hat{e}^N & \beta^a+\gamma^a\hat{e}^N\\
    -\beta^b-\gamma^b\hat{e}^N & 0
  \end{pmatrix}.
\end{equation}

The decomposition in Eq.~\eqref{Nconnection} adds new metric-independent fields. The 0-form $\alpha^{ab}$ is an antisymmetric tensor of spin-1. The 1-form $\beta^a = \beta^a{}_\mu \de{x}^\mu$ generically adds a spin-$2$, a spin-$1$ and a spin-$0$ fields. The last piece, $\gamma^a$, is a vector $0$-form of spin-1.

\subsection{Curvature and torsion}

The $N$-dimensional Lorentz curvature and torsion are given by the Cartan structure equations
\begin{align}
  \label{curvadef}
  \hat{R}^{AB} &= \mbox{d}\hat{\omega}^{AB}+\hat{\omega}^A_{\ \ C}\wedge\hat{\omega}^{CB} = \frac{1}{2} \hat{R}^{AB}{}_{CD} \, \hat{e}^C \wedge \hat{e}^D,\\
  \label{tordef}
  \hat{T}^A &= \mbox{d}\hat{e}^A+\hat{\omega}^A_{\ \ B}\wedge\hat{e}^B = \frac{1}{2} \hat{T}^{A}{}_{BC} \, \hat{e}^B \wedge \hat{e}^C. 
\end{align}
Using the definition of curvature in Eq.~\eqref{curvadef},  with the KK ansatz for the spin connection~\eqref{Nconnection}, we find
\begin{align}
  \label{R1}
  \hat{R}^{ab}&=R^{ab}+\sqrt{\phi}\alpha^{ab}F-\beta^a\wedge\beta^b\notag\\
  &+\left(\mbox{D}\alpha^{ab}+\frac{1}{2}\alpha^{ab}\mbox{d}\ln\phi-2\beta^{[a}\gamma^{b]}\right)\wedge\hat{e}^N,\\
  \label{R2}
  \hat{R}^{Na}&=-\left(\mbox{D}\beta^a+\sqrt{\phi}\gamma^a F\right)\notag\\
  &+\left(\alpha^a_{\ b}\beta^b-\mbox{D}\gamma^a-\frac{1}{2}\gamma^a\mbox{d}\ln\phi\right)\wedge\hat{e}^N.
\end{align}

Similarly from the definition of torsion in Eq.~\eqref{tordef}, we find its distinctive parts to be
\begin{align}\label{T1}
  \hat{T}^a &= T^a+\left(\beta^a-\alpha^a_{\ \ b}e^b\right)\wedge\hat{e}^N,\\
  \label{T2}
  \hat{T}^N &= \sqrt{\phi}F-\beta_b\wedge e^b+\left(\frac{1}{2}\mbox{d}\ln\phi+\gamma_be^b\right)\wedge\hat{e}^N.
\end{align}

\subsection{Bianchi identities\label{sec:bianchi}}

Considering the Bianchi identities for the KK structure described above, we find relevant information about the new fields. Taking the exterior covariant derivative over the $N$-dimensional curvature and torsion, the Bianchi identities are
\begin{align*}
  \hat{\text{D}}\hat{R}^{AB} &= 0 &\mbox{and}& &\hat{\text{D}}\hat{T}^A &= \hat{R}^A_{\ \ B}\wedge\hat{e}^B.
\end{align*}
A careful decomposition of the first one into its distinctive parts, gives the Bianchi identity for the curvature of the reduced spacetime, together with the second derivative rules
\begin{align*}
  \text{D}R^{ab} &=0, & \text{D}\mbox{D}\alpha^{ab} &=R^a_{\ \ l}\alpha^{lb}+R^b_{\ \ l}\alpha^{al},\\
  \text{D}\text{D}\beta^a &= R^a_{\ \ b}\beta^{b}, & \text{D}\text{D}\gamma^a &= R^a_{\ \ b}\gamma^{b}.
\end{align*}
The second Bianchi identity, gives nothing but its equivalent for the manifold $M_{N-1}$, this is
\begin{equation*}
  \mbox{D}T^a=R^a_{\ \ b}\wedge e^b.
\end{equation*}

This is taken as a proof of the tensorial nature of these new fields under the four-dimensional Lorentz transformations.

\section{Five-Dimensional Lovelock--Cartan Reduction\label{5EGB}}

The most general theory requiring the Lagrangian to be (i)~an invariant $N$-form under local Lorentz transformations; (ii)~a local polynomial of the vielbein, the Lorentz connection and their exterior derivatives; (iii)~constructed without the Hodge dual,\footnote{The Hodge dual maps $p$-forms into $(N-p)$-forms through $\star\left(\hat{e}^{A_1}\wedge ... \wedge\hat{e}^{A_p}\right) = \frac{1}{(N-p)!}\hat{\epsilon}^{A_1\ldots A_p}{}_{A_{p+1}...A_N}\,\hat{e}^{A_{p+1}}\wedge ... \wedge\hat{e}^{A_N}$.} is the Lovelock--Cartan theory of gravity~\cite{Mardones:1990qc}. This is the natural generalization of Lanczos--Lovelock when torsional degrees of freedom are present. 

In five dimensions, the action principle of such a theory is given by  
\begin{widetext}
  \begin{equation}
    \label{action5EGB}
    I = \int_{M_5} \hat{\epsilon}_{ABCDE} \Big(\frac{\alpha_0}{5}\hat{e}^A\wedge\hat{e}^B\wedge\hat{e}^C\wedge
    \hat{e}^D\wedge\hat{e}^E
    +\frac{\alpha_1}{3}\hat{R}^{AB}\wedge\hat{e}^C\wedge\hat{e}^D\wedge\hat{e}^E
    +\alpha_2\hat{R}^{AB}\wedge\hat{R}^{CD}
    \wedge\hat{e}^E\Big),
  \end{equation}
  where $\alpha_0$, $\alpha_1$ and $\alpha_2$ are dimensionful coupling constants. Its variation with respect to the five-dimensional vielbein and spin connection give the equations
  \begin{align}
    \label{delta e}
    \hat{\epsilon}_{ABCDE}\Big(\alpha_0\hat{e}^A\wedge\hat{e}^B\wedge\hat{e}^C\wedge\hat{e}^D
    + \alpha_1\hat{R}^{AB}\wedge\hat{e}^C\wedge\hat{e}^D
    + \alpha_2\hat{R}^{AB}\wedge\hat{R}^{CD}\Big)&=0,
    \\
    \label{delta w}
    \hat{\epsilon}_{ABCDE}\Big(\alpha_1\hat{e}^C\wedge\hat{e}^D+
    2\alpha_2\hat{R}^{CD}\Big)\wedge\hat{T}^E&=0.
  \end{align}
\end{widetext}

The last term in Eq.~\eqref{action5EGB} is the five-dimensional Gauss--Bonnet term for a Lorentz curvature written in exterior forms, which is analogue to Eq.~\eqref{GB} for the Riemannian case.

The Lovelock--Cartan theory in five dimensions allows a unique torsional extension which is not in the Lovelock series~\cite{Mardones:1990qc},
\begin{align}
  \label{boundary}
  \mathcal{L}_{T} \propto \hat{T}_A\wedge \hat{R}^A_{\ B}\wedge\hat{e}^B.
\end{align}
However, it can be written as a boundary term, adding no dynamics to the field equations.

In this work, we will focus on the region in the parameter space where
\begin{equation}\label{delta}
  \Delta\equiv\alpha_1^2-4\alpha_0\alpha_2\neq 0.
\end{equation}
This condition place us outside the Chern--Simons point. Otherwise, the field equations~\eqref{delta e} and~\eqref{delta w} are invariant under a larger gauge group ($AdS_5$), while the action~\eqref{action5EGB} becomes the Chern--Simons form for that group~\cite{Zanelli:2005sa,Troncoso:1999pk}. 

In terms of the KK ansatz presented in the previous section, the field equations can be decomposed into its distinctive parts, regarding
its \(M_{4} \times S^1\) decomposition. The Eq.~\eqref{delta e} leads to 
\begin{align}\label{equation}
  \epsilon_{abcd}\Big[\alpha_0 e^b\wedge e^c\wedge e^d + \frac{1}{2}\alpha_1\left(M^{bc}-L^b\wedge e^c\right)\wedge e^d&\notag\\
    -\alpha_2\left(L^b\wedge M^{cd} + K^b\wedge N^{cd}\right)\Big]&=0,\\
  \epsilon_{abcd}K^b\wedge\left(\alpha_1e^c\wedge e^d + 2\alpha_2M^{cd}\right)&=0,\\
  \epsilon_{abcd}\big(\alpha_0 e^a\wedge e^b\wedge e^c\wedge e^d&\notag\\
  +\alpha_1M^{ab}\wedge e^c\wedge e^d+\alpha_2M^{ab}\wedge M^{cd}\big)&=0,\\
  \epsilon_{abcd}\left(\alpha_1N^{ab}\wedge e^c\wedge e^d+2\alpha_2N^{ab}M^{cd}\right)&=0,
\end{align}
while Eq.~\eqref{delta w} gives
\begin{align}
  \epsilon_{abcd}\Big[\alpha_1\left(e^c\wedge e^d\wedge Z-2e^c\wedge T^d\right)
    +2\alpha_2\Big(N^{cd}\wedge W&\notag\\
    +M^{cd}\wedge Z+2L^c\wedge T^d
    +2K^c\wedge V^d\Big)\Big]&=0,\\
  \epsilon_{abcd}\left[\alpha_1e^c\wedge e^ d\wedge W+2\alpha_2\left(M^{cd}\wedge W+2K^c\wedge T^d\right)\right]&=0,\\
  \epsilon_{abcd}\left(\alpha_1e^b\wedge e^c+2\alpha_2M^{bc}\right)\wedge T^d&=0,\\
  \epsilon_{abcd}\left[\alpha_1e^b\wedge e^c\wedge V^d+2\alpha_2\Big(M^{bc}\wedge V^d+N^{bc}\wedge T^d\Big)\right]&=0.
\end{align}
The fields $M^{ab}$, $N^{ab}$, $L^a$, $K^a$, $W$, $V^a$ and $Z$ are defined from the Eq.~\eqref{R1} to~\eqref{T2}, such that
\begin{align*}
  \hat{R}^{ab}&=M^{ab}+N^{ab}\wedge\hat{e}^5, & \hat{R}^{5a}&=K^a+L^a\wedge\hat{e}^5,\\
  \hat{T}^a&=T^a+V^a\wedge\hat{e}^5, & \hat{T}^5&=W+Z\wedge\hat{e}^5.
\end{align*}


\section{Dimensionally Reduced Lovelock--Cartan Cosmology\label{cosmos}}

\subsection{Cosmological ansatz}

In order to look for cosmological solutions, we demand the symmetries assumed by the cosmological principle, i.e. isotropy and homogeneity of the involved fields. This can be achieved by imposing the Lie derivative of each field along the Killing vectors which generates the symmetries, to vanish. Appendix~\ref{homotropic} is devoted to the details of how to find the general ansatz.

The four-dimensional vielbein compatible with the Friedmann--Robertson--Walker metric is determined only by one function, the scale factor $a(t)$, and their components read
\begin{align}
  \label{vielbein cosmo}
  e^0&=\mbox{d}t, & e^1&=\frac{a(t)}{\sqrt{1-kr^2}}\mbox{d}r,\\
  e^2&=a(t)r\mbox{d}\theta, & e^3&=a(t)r\sin\varphi\mbox{d}\varphi, 
\end{align}
where $k=+1,0,-1$ determines the extrinsic curvature of the four-dimensional manifold to be closed, flat or open, respectively. 

The scalar field is a time-dependent function, 
\begin{equation}
  \phi=\phi(t),
\end{equation}
which is interpreted as the scale factor of the compact extra dimension.

On the other hand, the spin connection adds two functions that we called $\omega(t)$ and $f(t)$,
\begin{align}
  \omega^{0i}&=\omega(t) e^i,\\
  \omega^{12}&=-\frac{\sqrt{1-kr^2}}{a(t)r}e^2-f(t)e^3,\\
  \omega^{13}&=-\frac{\sqrt{1-kr^2}}{a(t)r}e^3+f(t)e^2,\\
  \omega^{23}&=-\frac{\cot\theta}{a(t)r}e^3-f(t)e^1.
\end{align}

The nonvanishing components of the spin connection induced by the compact manifold are
\begin{align}
  \beta^0=-b(t)e^0,&\ \beta^i=\beta(t)e^i,\\
  \label{gamma cosmo}
  \gamma^0&=-\gamma(t).
\end{align}

The $1$-form $A$ is $A(t)\text{d}t$, has vanishing field strength $F$. Thus, without loss of generality, it can be fixed to zero by means of an $U(1)$ gauge transformation or, equivalently, a diffeomorphism transformation along the $z$-direction.

The equations of motion for the cosmological ansatz give a system of differential equations for the time-dependent functions defined above. Curvature and torsion for this ansatz can be seen in Appendix~\ref{homotropic}. We find that the only non-Riemannian branch demands  $\beta^a=0$. Thus $b(t)=\beta(t)=0$. Otherwise, the system is consistent only at the Chern--Simons point, where it degenerates. 

The final set of equations is the following:
\begin{align}
  \label{eqn1}
  h\left[\alpha_1+2\alpha_2\left(\omega^2+\frac{k}{a^2}-f^2\right)\right]+4\alpha_2f\left(\dot{f}+Hf\right)&=0,\\
  2\alpha_0+\alpha_1\left(\dot{\omega}+2\omega^2+\frac{k}{a^2}-f^2\right)\notag\\
  +2\alpha_2\left(\dot{\omega}+\omega^2\right)\left(\omega^2+\frac{k}{a^2}-f^2\right)&=0,\\
  2\alpha_2\left(\frac{\dot{\phi}}{2\phi}+\gamma\right)f\left(\dot{f}+Hf\right)
  +h^2\left(\alpha_1-2\alpha_2\omega\gamma\right)&=0,\\
  \omega\left(\frac{\dot{\phi}}{2\phi}+\gamma\right)+\dot{\gamma}+\frac{\dot{\phi}}{2\phi}\gamma-\frac{\alpha_1}{2\alpha_2}&=0,\\
  \left(\omega^2+\frac{k}{a^2}-f^2\right)\left(\alpha_1-2\alpha_2\omega\gamma\right)-\alpha_1\omega\gamma+2\alpha_0&=0,\\
  \label{eqn2}
  \left(\dot{\omega}+\omega^2\right)\left(\alpha_1-2\alpha_2\omega\gamma\right)-\alpha_1
  \omega\gamma+3\alpha_0-\frac{\alpha^2_1}{4\alpha_2}&=0,
\end{align}
where dot stands for time-derivative. For the sake of simplicity, we have defined $h(t)=\omega(t)-H(t)$, where $H=\dot{a}/a$ is the Hubble function.

\subsection{Solutions}

The system is integrated and develops two non-Riemannian branches, one for each value of the parameter $u_\pm$ defined to be
\begin{equation}\label{u}
  u_\pm=\frac{2\alpha_1\pm\sqrt{6\Delta}}{4\alpha_2},
\end{equation}
where $\Delta$ was given in Eq.~\eqref{delta}.

The Eqs.~\eqref{eqn1} to~\eqref{eqn2} are reduced to the Riemannian system when \mbox{$f=h=0$} and \mbox{$\gamma=-\dot{\phi}/2\phi$}. The details of such a model can be seen in Refs.~\cite{Deruelle:1986iv,Deruelle:2003ck,Henriques:1986jw,Ishihara:1986if,Kleidis:1997mu}.

Due to Eq.~\eqref{u}, the solutions will be valid in the region of the parameter space where $\Delta>0$. The function $\omega(t)$ satisfies the equation
\begin{align}
  \dot{\omega}+\omega^2+u_{\pm} &= 0,
\end{align}
which, for the three significantly different values of $u_\pm$, has the following solutions
\begin{align}
  u_\pm>0&:\omega(t)=-\sqrt{u_\pm}\tan\left[\sqrt{u_\pm}\left(t-t_0\right)\right],\\
  u_\pm=0&:\omega(t)=\left(t-t_0\right)^{-1},\\
  u_\pm<0&:\omega(t)=\sqrt{-u_\pm}\tanh\left[\sqrt{-u_\pm}\left(t-t_0\right)\right],
\end{align}
where $t_0$ is an integration constant to be fixed. We express the time-dependence of the remaining fields in terms of $\omega$ and list them explicitly in Appendix~\ref{solutions t}. The solutions read
\begin{align}
  \gamma(\omega)&=\frac{u_\pm}{\omega},\\
  \phi(\omega)&=\frac{\phi_0\,\omega^2}{\omega^2+u_\pm}\exp\left[\frac{\mp\sqrt{6\Delta}}{4\alpha_2\left(\omega^2+u_\pm\right)}\right],\\
  a(\omega)&=\frac{a_0}{\sqrt{\omega^2+u_\pm}}\exp\left[\frac{\pm\sqrt{6\Delta}} {24\alpha_2\left(\omega^2+u_\pm\right)}\right],\\
  f^2(\omega)&=\left(\omega^2+u_\pm\right)\frac{k}{a_0^2}\exp\left[\frac{\mp\sqrt{6\Delta}}{12\alpha_2\left(\omega^2+u_\pm\right)}\right]\notag\\
  &+\omega^2  + \frac{3\alpha_1\pm\sqrt{6\Delta}}{6\alpha_2},
\end{align}
where $a_0$ and $\phi_0$ are integration constants.

Hereon, we will consider $\alpha_1$ to be positive,\footnote{In all the plots, we fixed $\alpha_1 = 1$.} because this parameter admits the interpretation of the Newton's gravitational constant. It is worth mentioning the existence of models of gravity which are free of the linear curvature term. These models are refer to as Pure Lovelock (PL) gravities~\cite{Cai:2006pq} and consider the cosmological term and the polynomial of highest order in the curvature. Even though the analysis that follows will consider a positive non-zero value for $\alpha_1$, the solutions listed above are also valid in the regime where $\alpha_1=0$, that constitutes the non-Riemannian extension of PL theories.
\begin{figure}[H]
  \includegraphics[width=\linewidth]{cond_u_pos.pdf}
  \caption{Allowed regions, in the $\alpha_0$--$\alpha_2$ plane, for $u_+$ and $u_-$ to be positive defined.}
  \label{cond_u_pos}
\end{figure}


\subsubsection{Bouncing solutions}

In Fig.~\ref{cond_u_pos} we show the allowed region, in the $\alpha_0$--$\alpha_2$ parameter space, for $u_\pm$ to be positive. 


All the solutions for $u_\pm>0$ are periodic, as shown in the behavior of the scale factor in Fig.~\ref{a_u_pos}. In that case, the scale factor $a(t)$ starts from a singular point and reaches a future one after a time $t_{\rm bounce}=\pi/\sqrt{u_\pm}$. Depending on the particular region on the parameter space, the scale factor undergoes an expanding and contracting age, allowing a intermediate bounce without collapsing, before a big crunch. This is the case of the third curve in Fig.~\ref{a_u_pos}. Otherwise, it expands and collapses in a {simple} oscillatory way.
%\hl{One can also notice from the plot, that solutions parametrized by $u_+$ do not present inflationary phase, while those parametrized by $u_-$ do---the scale factor grows exponentially for a certain interval of time. Interestingly, the solutions with inflationary phases collapse in a damped form.}
Moreover, the bouncing behavior of the extra dimension is given by the scalar field in a phase difference of $\pi/2$ with respect to $a(t)$, as shown in Fig.~\ref{phi_u_pos}. It is worth to point that the {intermediate bounce} present in the scale factor is much prominent in the modulation of the extra dimension.
\begin{figure}[H]
  \includegraphics[width=.95\linewidth]{a_u_pos.pdf}
  \caption{Behavior of the scale factor, \(a(t)\), as a function of the scaled time, $\tau$, for $u_\pm > 0$.}
  \label{a_u_pos}
\end{figure}
\begin{figure}[H]
  \includegraphics[width=.95\linewidth]{phi_u_pos.pdf}
  \caption{Behavior of the $\phi$ field as a function of the scaled time, $\tau$, for $u_\pm > 0$.}
  \label{phi_u_pos}
\end{figure}

\subsubsection{Expanding and contracting solutions}

None of the other cases, $u_- = 0$ and $u_\pm < 0$, present bouncing behavior.
The allowed parameter space for $u_+$ and $u_-$ to be negative is shown in Fig.~\ref{cond_u_neg}, while the restriction $u_- = 0$ sets $\alpha_2 = \tfrac{1}{12 \alpha_0}$.
Nonetheless, depending on the values of the parameters there are three distinctive conducts: (i)~eternal expansion, (ii)~eternal contraction, and (iii)~initial expansion with a final collapse.

\paragraph{The case $u_- = 0$} has two behaviors depending on whether $\alpha_0$ is positive or negative.

As shown in Fig.~\ref{a_u_0}, for positive $\alpha_2$ the solutions fit into the third category above, while for negative $\alpha_2$ the solutions expand eternally. Notice that in the former case the Universe's fate is an asymptotic collapse, whilst in the latter expands forever.
\begin{figure}[H]
  \includegraphics[width=\linewidth]{a_u-_0.pdf} 
  \caption{Distinctive behaviors of the scale factor for $u_- = 0$.}
  \label{a_u_0}
\end{figure}
\begin{figure}[H]
  \includegraphics[width=\linewidth]{phi_u-_0.pdf}
  \caption{Distinctive behavior of the $\phi$ field for $u_- = 0$.}
  \label{phi_u_0}
\end{figure}
The $\phi$ field (see Fig.~\ref{phi_u_0}) grows infinitely for $\alpha_2 > 0$, or asymptotically goes to zero for $\alpha_2 < 0$. This latter behavior provides a {dynamical} compactification, which might serve as a mechanism to assure---at certain time---the decoupling of the zeroth mode from the Kaluza--Klein tower.


\paragraph{The case $u_\pm < 0$} has solutions which either expand or collapse eternally, as shown in Fig.~\ref{a_u_neg}. The typical evolution of the Universe with $u_+ > 0$ is to grow infinitely, while for negative $u_-$ the expansion (contraction) corresponds to $\alpha_2$ positive (negative). Figure~\ref{phi_u_neg}, shows that the asymptotic {size} of the extra dimension, modulated by the $\phi$ field, is approximately reciprocal to the behavior of the scale factor but it is not monotonic.
%% , depending on the sign
%% of $\alpha_2$ and the $\pm$-sign in $u_\pm$, the scale factor will
%% infinitely expand while the scalar field reaches an asymtotic
%% vanishing value for late cosmic times, after a period of
%% expansion. This is interpreted as the late time behavior of the
%% radius of the compact manifold $S^1$, which for this family of
%% solutions provides a natural explanation for its scale. Otherwise, in
%% contrast to the previous behavior, the scale factor contracts to an
%% asymtotic vanishing value while the scalar field has no upper
%% bound.
For all these solutions the scale factor remains finite at $t = 0$, and they do not present an initial singularity. It has been reported that the Gauss--Bonnet term can prevent the Universe to expand from an initial singularity~\cite{Deruelle:1986iv,Henriques:1986jw,Ishihara:1986if}, which is also this case.

\begin{figure}[H]
  \includegraphics[width=\linewidth]{cond_u_neg.pdf}
  \caption{Allowed regions, in the $\alpha_0$--$\alpha_2$ plane, for $u_+$ and $u_-$ to be positive defined.}
  \label{cond_u_neg}  
\end{figure}

\begin{figure}[H]
  \includegraphics[width=.95\linewidth]{a_u_neg.pdf}
  \caption{Different behaviors for the scale factor, as function of the scaled time $\tau$, for $u_\pm < 0$.}
  \label{a_u_neg}
\end{figure}

%\pagebreak
\begin{figure}[H]
  \includegraphics[width=.95\linewidth]{phi_u_neg.pdf}
  \caption{Different behaviors for the $\phi$ field, as function of the scaled time $\tau$, for $u_\pm < 0$.}
  \label{phi_u_neg}
\end{figure}


\subsection{Effective energy density and pressure}

The Eq.~\eqref{equation} can be split into its Riemannian and non-Riemannian parts, by means of the identity~\eqref{curvature decomp}, taking the familiar form
\begin{align}\label{einstein equation}
  -\frac{1}{2}\epsilon_{abcd}\left(\tilde{R}^{bc} - \frac{\Lambda}{3}\,e^b\wedge e^c\right)e^d &= \kappa_{G}\,\tau^{\mathrm{eff}}_a,
\end{align}
with $\Lambda \equiv -6\alpha_0/\alpha_1$, $\kappa_{G} \equiv 2/\alpha_1$, and the effective energy-momentum $3$-form defined as
\begin{align}
  \label{taua}
  \tau_a^{\mathrm{eff}}&= \tfrac{1}{2}\epsilon_{abcd}\Big[\tfrac{1}{2}\alpha_1\left(\kappa^b_{\ l}\wedge\kappa^{lc}+\tilde{\text{D}}\kappa^{bc}\right)\wedge e^d\\
    & \quad +\left(\text{D}\gamma^b+\tfrac{1}{2}\gamma^b\text{d}\ln\phi\right)\wedge\left(\tfrac{1}{2}\alpha_1e^c\wedge e^d+\alpha_2R^{cd}\right)\Big], \notag
\end{align}
where $\kappa^{ab}$ is the contorsion $1$-form defined as the torsional correction to the spin connection (see Eq.~\eqref{spin separation}).
The contributions of $F$, $\alpha^{ab}$ and $\beta^a$ have not been taken into account since, for all practical purposes of this article, they vanish.
%% The left hand side of the Eq.~\eqref{einstein equation} is the Einstein $3$-form, which is dual to the Einstein tensor with cosmological constant in four dimensions.
The left hand side of Eq.~\eqref{einstein equation} is the 3-form which, in components, yields the four-dimensional Einstein equations with cosmological constant.
We identify the right hand side as an effective energy-momentum $3$-form, which behaves as a perfect fluid. It contains the torsional and the higher dimensional degrees of freedom.

The energy density $\rho$ and the pressure $p$ are obtained through the identities
\begin{align}
  \tau_0^{\mathrm{eff}} &= -\frac{1}{3!} \, \rho \, \epsilon_{0ijk} \; e^i\wedge e^j\wedge e^k,\\
  \tau_i^{\mathrm{eff}} &= -\frac{1}{2} \,p \, \epsilon_{0ijk} \; e^0\wedge e^j\wedge e^k,
\end{align} 
giving
\begin{align}
  \label{rho}
  \rho &= -\frac{3}{2}\alpha_1\left(h^2-f^2+2Hh\right) \\
  & \quad + 3\omega\gamma\left[\frac{1}{2}\alpha_1+\alpha_2\left(\omega^2+\frac{k}{a^2}-f^2\right)\right], \notag
  \\
  \label{presion}
  p &= \alpha_1\left[\dot{h}+Hh+\frac{1}{2}\left(h^2-f^2+2Hh\right)\right]\\
  & \quad - 2\omega\gamma\left[\frac{1}{2}\alpha_1+\alpha_2\left(\dot{\omega}+H\omega\right)\right]\notag\\
  & \quad +\left(\dot{\gamma}+\frac{1}{2}\gamma\dot{\Phi}\right)\left[\frac{1}{2}\alpha_1+\alpha_2\left(\omega^2+\frac{k}{a^2}-f^2\right)\right]. \notag
\end{align}
Using the solutions obtained in the previous section, these expressions are
\begin{align}
  \label{our-rho}
  \rho &= \frac{3}{2}\alpha_1\left(H^2 + \frac{k}{a^2}\right) + 3\alpha_0, \\
  \label{our-p}
  p &= -\alpha_1\left(\dot{H} + \frac{3}{2}\,H^2 + \frac{k}{2a^2}\right) - 3\alpha_0,
\end{align}
and satisfy the continuity equation $\dot{\rho}+3H\left(\rho+p\right)=0$. 

For $u_\pm<0$, the energy density remains finite at the begining. From the Eqs.~\eqref{our-rho} and~\eqref{our-p} one can see that the induced energy density and pressure are not positive definite quantities. In fact, the Universe undergoes through an accelerating expansion phase due to the presence of torsion and the extra dimensional fields. 

%_-----------------------------------------------------------------------------------------

\section{Discussion and Conclusions\label{conclusions}}

In this paper we have presented the dimensional reduction of the five-dimensional Lovelock--Cartan theory introduced in Ref.~\cite{Mardones:1990qc}, under the assumption that the compact dimension has $S^1$ topology. An interesting feature of these theories is that, unlike the Einstein--Cartan theory, torsion is allowed to propagate.

In the generic reduction, the effective theory has a spin-2 particle, a spin-1 $U(1)$ gauge boson, and a spin-0 scalar particle coming from the decomposition of the metric, while the decomposition of the spin connection introduces three extra fields: two spin-1 particles, $\alpha^{ab}$ and $\gamma^a$, and the 1-form $\beta^a$ yields a new spin-2, a spin-1 and a spin-0 particles. These new fields do not transform under the gauge group (see Sec.~\ref{sec:bianchi}).

We used the most general ansatz compatible with the cosmological symmetries (see Appendix~\ref{homotropic}) to find solutions of the Friedmann--Robertson--Walker class in the effective theory. The equations of motion [Eqs.~\eqref{eqn1}--\eqref{eqn2}] ensures that solutions with nontrivial torsion have vanishing $\alpha^{ab}$ and $\beta^a$.

The solutions are parametrized by a variable $u_\pm$, related to the fundamental couplings of the theory through the Eq.~\eqref{u}, and exhibit three different sectors depending on whether this parameter is positive, negative or zero.

The behavior of the Universes described by our solutions is four-fold: (i)~eternal expansion, (ii)~eternal contraction, (iii)~initial expansion with a final collapse, and (iv)~bouncing. We show the typical evolution of the scale factor, for each case, in Figs.~\ref{a_u_pos}, ~\ref{a_u_0} and~\ref{a_u_neg}. The size of the extra dimension is dynamical, and there are sectors in the parameter space where there exists an asymptotic dynamical mechanics of compactification (see Figs.~\ref{phi_u_0} and~\ref{phi_u_neg}). 

After the decomposition of the curvature into Riemannian and non-Riemannian parts, the Eq.~\eqref{equation} takes the form of the usual Einstein's equations where the extra fields, coming from the extra dimensional and torsional degrees of freedom, originate an effective energy-momentum tensor. It is important to mention that this tensor is nonstandard, in the sense that it inherit a coupling between matter and curvature---see Eq.~\eqref{taua} for the definition---, due to the non-minimal coupling among the four-dimensional gravity and the induced fields. Moreover, the energy density and the pressure are not positive definite, which allows us to interpret these induced matter as exotic, providing a playground for finding gravitational explanation to dark matter/energy.

%% It is known that KK theory, in its simplest version, gives rise to matter fields as a manifestation of the extra dimensional metric degrees of freedom. The kind of matter arising in the reduced spacetime is subordinated to the topology taken for the extra compact manifold. In general, those fields will gravitate and modulate the behavior of spacetime in its time evolution. 

%% We find cosmological solutions of the Friedmann--Robertson--Walker class for the dimensionally reduced Lovelock--Cartan theory in five dimensions~\cite{Mardones:1990qc}. This theory allows torsion to propagate in vacuum, in contrast to the Einstein--Cartan theory. The induced matter fields in the reduced cosmological scenario are the usual scalar field --a metric degree of freedom-- and a vector field, coming from the dimensional reduction of a general spin connection. Those fields participate in the effective energy-momentum form. Neither the energy density nor the presure are strictly positive definite, therefore, can be interpreted as a class of matter of exotic features. Five-dimensional vacuum solutions are studied, where accelerating, bouncing and inflationary solutions were found.

There are periods of time where $f^2(t)$ becomes negative. As we can see from Eq.~\eqref{homotorsion}, $f$ corresponds to the completely antisymmetric part of torsion and is unseen by classical particles following geodesics. It does not couple minimally to spin-0 or spin-1 bosons, but it does appear as an effective interaction term in the Dirac Lagrangian written in a curved spacetime~\cite{Hehl:1976kj}
\begin{align*}
  \mathcal{L}_{\text{Int}}&=-\frac{i}{8}T_{\alpha\mu\nu}\bar{\Psi}\Gamma^{\alpha\mu\nu}\Psi =\frac{3}{2}f\bar{\Psi}\Gamma^0\Gamma^5\Psi.
\end{align*}
It couples to the chiral current. For imaginary values of $f(t)$, the corresponding violation of current conservation can be interpreted as particle creation. Similar results where found in~\cite{Toloza:2013wi} for other cosmological model with torsion which shares some of the couplings of a reduced action.

\begin{acknowledgments}
  The authors would like to thank A.~R.~Zerwekh, I.~Schmidt, N.~A.~Neill, C.~O.~Dib, O.~Miskovic, A.~Toloza and J.~Zanelli for enlightening discussions on this work. O.~C-F. has been partially supported by the PAI project No.~79140040 (CONICYT--Chile). The Centro Cient\'ifico y Tecnol\'ogico de Valpara\'iso (CCTVal) is funded by the Chilean Goverment through the Centers of Excellence Base Financing Program of CONICYT.
\end{acknowledgments}
%_-----------------------------------------------------------------------------------------


\appendix

\section{Riemann--Cartan Geometry\label{Riemann-Cartan}}

In Riemann--Cartan geometry, both the vielbein $e^A$ and the spin connection $\omega^{AB}$ are independent features of the manifold. We will drop the hats since we are not dealing with the KK decomposition. The spin connection, however, can be decomposed in a Riemannian part, which is metric dependent, $\tilde{\omega}^{AB}$ satisfying $\tilde{\mbox{D}}e^A=0$, and a contorsion piece $\kappa^{AB}=-\kappa^{BA}$, such that 
\begin{equation}\label{spin separation}
  \omega^{AB}=\tilde{\omega}^{AB}+\kappa^{AB}.
\end{equation} 
Therefore, the torsion 2-form defined as the covariant derivative of the vielbein with respect to the total spin connection, is
\begin{equation}
  T^A=\kappa^A_{\ B}\wedge e^B.
\end{equation}
On the other hand, the curvature $2$-form also suffers corrections with respect to the Riemannian one due to the presence of torsional degrees of freedom. This can be seen explictly by taking the definition of curvature in Eq.~\eqref{curvadef} and~\eqref{spin separation} to find
\begin{equation}\label{curvature decomp}
  R^{AB} = \tilde{R}^{AB} + \tilde{\mbox{D}}\kappa^{AB} + \kappa^A_{\ C}\wedge\kappa^{CB},
\end{equation}
where $\tilde{R}^{AB}$ is the Riemannian curvature constructed with $\tilde{\omega}^{AB}$.


\section{Isotropic--Homogeneus Ansatz\label{homotropic}}

The cosmological principle demands the spatial section of spacetime to be isotropic and homogeneus. This means that the fields involved in the model must be compatible with this assumption. A spacetime is isotropic with respect to certain point $P$ if after a rotation with respect to an axis passing through $P$, all the geometrical properties remain invariant, thus the spacetime looks the same in all directions. Homogeneity is understood such that spacetime looks the same from every point $P$. These two assumptions are translated in the Killing equations, which are the vanishing of the Lie derivatives of the fields along the vectors which generate the symmetries $\{\zeta^\lambda_{i}\}$. In particular the Killing equations for the metric tensor and the torsion tensor are
\begin{align}
  \label{killing metric}
  \text{\textsterling}_i g_{\mu\nu}&=\zeta^\lambda_i\partial_\lambda g_{\mu\nu}+\partial_\mu\zeta^\lambda_i g_{\lambda\nu}+\partial_\nu\zeta^\lambda_i g_{\mu\lambda}=0,\\
  \text{\textsterling}_i T^\alpha_{\ \mu\nu}&=\zeta^\lambda_i\partial_\lambda T^\alpha_{\ \mu\nu}-\zeta^\alpha_i\partial_\lambda T^\lambda_{\ \mu\nu}
  +\zeta^\lambda_i\partial_\mu T^\alpha_{\ \lambda\nu}\notag\\
  &\quad +\zeta^\lambda_i\partial_\nu T^\alpha_{\ \mu\lambda}=0,
\end{align}
where $T^\alpha_{\ \mu\nu}$ are the components of the torsion 2-form defined by $T^a=\frac{1}{2}e^a_{\ \alpha}T^\alpha_{\ \mu\nu}\mbox{d}x^\mu\wedge\mbox{d}x^\nu$. The same must hold for the new fields
\begin{align}
  \label{killing vector}
  \text{\textsterling}_i A_\mu&=\zeta^\lambda_i\partial_\lambda A_\mu+\zeta^\lambda_i\partial_\mu A_\lambda=0,\\
  \text{\textsterling}_i \phi &=\zeta^\lambda_i\partial_\lambda\phi=0,
\end{align}
and also for the components of $\alpha_{\mu\nu}=-\alpha_{\nu\mu}$, $\beta_{\mu\nu}$ and $\gamma_\mu$, defined such that
\begin{align*}
  \alpha^{ab}&=E^{a\mu}E^{b\nu}\alpha_{\mu\nu},\\
  \beta^a&=E^{a\mu}\beta_{\mu\nu}\mbox{d}x^\nu,\\
  \gamma^a&=E^{a\mu}\gamma_\mu,
\end{align*} 
and whose Killing equations are analogous to the tensorial one,~\eqref{killing metric} and vectorial one,~\eqref{killing vector}.

The set of Killing vectors $\{\zeta^\lambda_i\}$ are the generators of $SO(3)$, which generate the spatial rotations in three dimensions, $\xi_i=\epsilon_{ijk}x_j\partial_k$, and the Killing vectors assosiated with spatial translations $\mathcal{P}_i=\sqrt{1-kr^2}\partial_i$, satisfy the algebra
\begin{align}
  \left[\xi_i,\xi_j\right]&=\epsilon_{ijk}\xi_k,\\
  \left[\mathcal{P}_i,\mathcal{P}_j\right]&=-k\epsilon_{ijk}\xi_k,\\
  \left[\xi_i,\mathcal{P}_j\right]&=\epsilon_{ijk}\xi_k.
\end{align}
These requirements on the fields are translated to a set of first order differential equations whose most general solutions determines our ansatz structure from Eq.~\eqref{vielbein cosmo} to \eqref{gamma cosmo}. The field strengths for isotropic--homogeneus vielbein and spin connection are written next. The curvature components are
\begin{align}
  R^{0i}&=\left(\dot{\omega}+H\omega\right)e^0\wedge e^i+f\omega\epsilon^i_{\ jk}e^j\wedge e^k,\\
  R^{ij}&=\left(\omega^2+\frac{k}{a^2}-f^2\right)e^i\wedge e^j
  -\left(\dot{f}+Hf\right)\epsilon^{ij}_{\ \ k}e^0\wedge e^k,
\end{align}
and the torsion components
\begin{equation}\label{homotorsion}
  T^0=0,\quad T^i=-he^0\wedge e^i+f\epsilon^i_{\ jk}e^j\wedge e^k,
\end{equation}
where $h(t)$ and $H(t)$ were defined in Section \ref{5EGB}.

\section{Time-dependence of the solutions\label{solutions t}}

We supply here the time-dependent expressions for the cosmological solutions, given the differents values of $u_\pm$, in terms of the dimensionless parameter~$\tau=\sqrt{|u_\pm|}(t-t_0)$.
\medskip

$\bullet\ u_\pm>0$:
\begin{align}
  \gamma(t)&=-\sqrt{u_\pm}\cot\tau,\\
  \phi(t)&=\phi_0\sin^2\tau\exp\left[\frac{\mp\sqrt{6\Delta}}{4\alpha_2 u_\pm}\cos^2\tau\right],\\
  a(t)&=\frac{a_0}{\sqrt{u_\pm}}|\cos\tau|\exp\left[\frac{\pm\sqrt{6\Delta}}{24\alpha_2 u_\pm}\cos^2\tau\right],\\
  f^2(t)&=u_\pm\frac{k}{a_0^2}\sec^2\tau\exp\left[\frac{\mp\sqrt{6\Delta}}{12\alpha_2u_\pm}\cos^2\tau\right]\notag\\
  &+u_\pm\tan^2\tau + \frac{3\alpha_1\pm\sqrt{6\Delta}}{6\alpha_2}.
\end{align}

$\bullet\ u_\pm=0$:
\begin{align}
  \gamma(t)&=0,\\
  \phi(t)&=\phi_0\exp\left[\frac{\mp\sqrt{6\Delta}}{4\alpha_2}(t-t_0)^2\right],\\
  a(t)&=a_0|t-t_0|\exp\left[\frac{\pm\sqrt{6\Delta}} {24\alpha_2}(t-t_0)^2\right],\\
  f^2(t)&=\frac{k}{a_0^2(t-t_0)^2}\exp\left[\frac{\mp\sqrt{6\Delta}}{12\alpha_2}(t-t_0)^2\right]\notag\\
  &+(t-t_0)^{-2} + \frac{3\alpha_1\pm\sqrt{6\Delta}}{6\alpha_2}.
\end{align}

$\bullet\ u_\pm<0$:
\begin{align}
  \gamma(t)&=-\sqrt{-u_\pm}\coth\tau,\\
  \phi(t)&=\phi_0\sinh^2\tau\exp\left[\frac{\mp\sqrt{6\Delta}}{4\alpha_2u_\pm}\cosh^2\tau\right],\\
  a(t)&=\frac{a_0}{\sqrt{-u_\pm}}\cosh\tau\exp\left[\frac{\pm\sqrt{6\Delta}} {24\alpha_2u_\pm}\cosh^2\tau\right],\\
  f^2(t)&=-u_\pm\frac{k}{a_0^2}\cosh^{-2}\tau\exp\left[\frac{\mp\sqrt{6\Delta}}{12\alpha_2u_\pm}\cosh^2\tau\right]\notag\\
  &-u_\pm\tanh^2\tau + \frac{3\alpha_1\pm\sqrt{6\Delta}}{6\alpha_2}.
\end{align}
%%%%%%%%% BIBLIOGRAPHY %%%%%%%%%
%%\bibliographystyle{apsrev4-1}
\bibliography{References.bib}

\end{document}

